\documentclass[18pt]{beamer}
\usepackage[T1]{fontenc}
\usepackage[protrusion=true,expansion=true]{microtype}
%usepackage[minionint,mathlf]{MinionPro}
\usepackage[sc,osf]{mathpazo}
%\renewcommand{\sfdefault}{Myriad-LF}

\usepackage{textcomp}
\usepackage{graphicx}
\usepackage{semantic}

\usetheme{default}
\usefonttheme{professionalfonts}
%\usefonttheme[]{structuresmallcapsserif}
\begin{document}

\frame{
  \begin{center}
    Formalization of JANUS\\
    \emph{or}\\
    ``Explaining your proof to the machine''\\
  \end{center}
}

\frame{
  \begin{center}
    \large{The work:}\\
    \begin{itemize}
    \item Take JANUS semantics from papers.
    \item Encode the semantics in Coq.
    \item Show that the properties of the papers hold.
    \end{itemize}
  \end{center}
}

\frame{
  \begin{center}
    Coq absolutely loves detail!\\
    32-bit numbers and stores need attention.
  \end{center}
}

\frame{
  \begin{center}
    32-bit number representation (Xavier Leroy et. al.)\\
    integer $z \in \mathbb{Z}$, and\\
    A proof of $0 \leq z < 2^{32}$.\\
    Key Lemma:\\
    $\forall z \in \mathbb{Z}, \quad 0 \leq z \mod 2^{32} < 2^{32}$
  \end{center}
}

\frame{
  \begin{center}
    Store representation:\\
    1st attempt:\\
    $\sigma \colon \mathrm{Var} \to \mathrm{w32}$\\
    \small{\texttt{Definition empty (\_ : var) := 0.}}\\
    2nd attempt:\\
    $\sigma \colon \mathrm{Var} \to \mathrm{w32}_{\perp}$\\
    \small{\texttt{Definition empty (\_ : var) := None.}}
  \end{center}
}

\frame{
  \begin{center}
    Expressions:\\
    $\sigma |- e => n$\\
    Inputs: $\sigma$ and $e$, outputs: $n$.\\
    Don't need a relation, this is a function!\\
    $denoteExp \colon \mathrm{store} \to \mathrm{exp} \to \mathrm{w32}$
  \end{center}
}

\frame{
  \begin{center}
    Statement assignments:\\
    $x \; +\!\!= e$
    \begin{itemize}
    \item $x$ must not occur in $e$. No formalization in the PEPM2007
      paper.
    \item 1st attempt: Drive a syntactic check (static semantics)
    \item 2nd attempt: Implement store hiding!
    \end{itemize}
  \end{center}
}

\frame[Stores]{
  \begin{center}
    Suppose $\sigma \colon \mathrm{Var} \to W^{32}_{\perp}$ is a
    store.\\
    Lookup of $x$ is $\sigma(x)$.\\
    Returns either $\lfloor v \rfloor$ if $x$ maps to $v$ or $\perp$
    if no such mapping exist.
    Extension $\sigma[x \mapsto v]$ is
    \begin{equation}
      \label{eq:1}
      \sigma[x \mapsto v](y) =
      \begin{cases}
        \lfloor v \rfloor &\text{if}\; x = y\\
        \sigma(y) & \text{otherwise}
      \end{cases}
    \end{equation}
  \end{center}
}

\frame[Stores]{
  \begin{center}
    Hiding notation $(\sigma \setminus x)(y)$. Semantics:
    \begin{equation}
      \label{eq:2}
      (\sigma \setminus x)(y) =
      \begin{cases}
        \perp &\text{if}\; x = y\\
        \sigma(y) & \text{otherwise}
      \end{cases}
    \end{equation}
  \end{center}
}

%% TODO: We need to supply examples of proofs for stores here.

%% TODO: Show how we cleverly can utilize the hiding feature in the
%% semantics.

%% Describe Janus1.
%% Show key properties we have shown about Janus1.

%% Show some simple proofs about the stores as an example.
%% Show forward determinism or something equally simple.

\frame{
  \begin{center}
    But how does it look in the real world?\\
    ProofGeneral, an emacs environment for proof assistants:\\
    \includegraphics[width=100px]{ProofGeneral}
  \end{center}
}


\end{document}

%%% Local Variables: 
%%% mode: latex
%%% TeX-master: t
%%% End: 

%%% Local Variables: 
%%% mode: latex
%%% TeX-master: t
%%% End: 
