\chapter{Extensions to C(co-)iC}

The \coq{} system uses the Calculus of (co-)inductive
constructions as its underlying programming language and thus (by
Curry-Howard) as its logic. This system is immensively powerful and
can encode large parts of traditional matematics. However, there are
certain properties which are unprovable in the C(co-)iC.

An unprovable property should not let an aspiring \coq{} user down. We
can just add that property as an axiom to the system and take it for
granted. We must be very careful not to add a property to the system
which makes the system inconsistent though. It is known,
\cite{coq-add} section 5.2, that one can add the properties of \emph{Proof
  irrelevance} and \emph{Extensionality} simultaneously without
introducing any inconsistencies.

\paragraph{Proof Irrelevance}
\label{sec:proof-irrelevance}

The Irrelevance of proofs state that if we have two ways of proving
the same thing, it is irrelevant which one we look at. They are
decided as being equal. Formally:
\begin{axm}
  Suppose $P$ is any proposition. Let $p_1$ and $p_2$ be proofs of
  proposition $P$. In \coq{} this means $p_1$ and $p_2$ are terms of
  type $P$. Then $p_1 = p_2$.
\end{axm}

\paragraph{Extensionality}
\label{sec:extensionality}

The extensionality axiom states that if two functions agree on all
values, then they are equal. Formally:
\begin{axm}
  Suppose we have $f, g \colon A \to B$. Then if $\forall x \in A,
  f(x) = g(x)$ we have $f = g$.
\end{axm}

%%% Local Variables: 
%%% mode: latex
%%% TeX-master: "master"
%%% End: 
