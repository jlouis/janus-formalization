\chapter{Choice of formalization framework}
\label{chap:thm-prover-choice}

In this chapter, we survey two theorem proving frameworks and chooses
one of them for the JANUS development.

\fixme{rewrite from here}
One might want to know what decided the formal verification framework
to use. The author has used two frameworks in the past, \coq{} and \twelf{}
and it felt logical to use one of those in the formal verification.

One has to make the choice of framework early on and then stick to
it. It is exactly like choosing a programming language for a given
task.

\coq{} is a proof assistant using an (co-)inductive variant of the
Calculus of Constructions (CoC). It relies on the Curry-Howard
correspondence to get a link between program/proof and
types/theorems. Proof are carried out by giving tactics; hints about
what to do next. These tactics then build up a term which constitutes
the proof via the Curry-Howard isomorphism. Finally, a low level type
checker makes sure the proof is correct.

\twelf{} uses a dependently typed lambda calculus constrained to
certain canonical forms. This calculus is called LF. The syntax,
semantics, and proofs are then all encoded in this LF-language. For
proof validity, separate checker then verifies certain properties on
the correctness of the proof, see \cite{harper+05:how-to-believe,
  harper+07:mechanizing}.

\coq{} supports proof automation where parts of proofs are done
automatically by the computer. However, most of the time carrying out
proofs are spent on things other than the proof structure: a
considerable amount of time is spent on figuring out how to conduct
the proof. Another time consumer is feeding the necessary prior
knowledge to the proof system for the development. You can only carry
out 32-bit addition if the system knows what a 32-bit
number is, and what addition means. Properties of JANUS relies on
properties of 32-bit addition, so this must be fed to the framework as
well.

\twelf{} does not currently support automation of proof. On the other
hand, it provides proof \emph{adequacy}: there is an isomorphism
between the proof in \twelf{} and proof on paper. In other words, any
proof can be transcribed to a standard paper proof. If one is
interested in how the proof is carried out, \twelf{} has a more direct
approach. \coq{} does not provide a proof term that is easily
digestible by a human reader.

The JANUS development needs 32-bit numbers. These were available for
\coq{}, but not for \twelf{}. One could have used classic integers of
infinite size however. \coq{} further had support for arithmetic in
libraries, whereas in \twelf{}, we would have to encode this
ourselves. On the other hand, \twelf{} has a simpler system and excels
at working with lambda-style binders through
Higher-order-abstract-syntax (HOAS). For \twelf{}, it would be
beneficial to encode the functions via HOAS, but we can't utilize HOAS
for the store where it would have been even more beneficial. \twelf{}
also has support for several mutual induction schemes. These are also
available in \coq{}, but are considerably harder to use. From \coq{}
version 8.2, some syntactic sugar has been added to make its
use easier, but it has not seen much testing yet.

It was deemed that it would be best to have easy access to 32-bit
numbers and hence \coq{} were chosen. In hindsight certain mutual
induction proofs would probably have been easier to carry out in
\twelf{} and to rely on arbitrary sized integers would not have been
detrimental to the demonstration of the findings in this report. The
author would probably have used \twelf{} should he do the task again.

\section{Extensions to C(co-)iC}

We now extend \coq{} with two useful axioms: the \coq{} system uses
the Calculus of (co-)inductive constructions as its underlying
programming language and thus (by Curry-Howard) as its logic. This
system is immensely powerful and can encode large parts of traditional
mathematics. However, there are certain properties which are
unprovable in the C(co-)iC.

An unprovable property should not let an aspiring \coq{} user down. We
can just add that property as an axiom to the system and take it for
granted. We must be very careful not to add a property to the system
which makes the system inconsistent though. It is known,
\cite{coq-add} section 5.2, that one can add the properties of \emph{Proof
  irrelevance} and \emph{Extensionality} simultaneously without
introducing any inconsistencies.

\paragraph{Proof Irrelevance}
\label{sec:proof-irrelevance}

The Irrelevance of proofs state that if we have two ways of proving
the same thing, it is irrelevant which one we look at. They are
equal. Formally:
\begin{axm}
  Suppose $P$ is any proposition. Let $p_1$ and $p_2$ be proofs of
  proposition $P$. In \coq{} this means $p_1$ and $p_2$ are terms of
  type $P$. Then $p_1 = p_2$.
\end{axm}

\paragraph{Extensionality}
\label{coqext:extensionality}

The extensionality axiom states that if two functions agree on all
values, then they are equal. Formally:
\begin{axm}
  Suppose we have $f, g \colon A \to B$. Then if $\forall x \in A,
  f(x) = g(x)$ we have $f = g$.
\end{axm}
Our development uses extensionality in many places. It is used to
determine equality of (mathematical) functions by discrimination of
each element in the function. Without this extension, we would have
needed another encoding of the JANUS store, as we shall see later
on.

%%% Local Variables: 
%%% mode: latex
%%% TeX-master: "master"
%%% End: 
