\chapter{\januso{}}
\label{cha:januso}

The \januso{} language adds several extra judgements to the \janusz{}
language to make it more in par with the full complete language
specification. Concretely, we add:
\begin{itemize}
\item 32-bit unsigned integers with wrap-around
\item Procedure calls
\end{itemize}
at which point we only miss arrays and loops from the full Janus
language.

To do this, we must first introduce 32-bit integers into \coq{} and
then use the formalization of these in the development of \januso{}.

\section{32-bit Integers}
\label{sec:32-bit-integers}

The integers chosen in JANUS are as usual machine words on a 32-bit
machine. Most traditional computers the machine have an upper limit
to what numbers it can represent. A 32-bit machine has $32$ bits at
its disposal so it can represent $2^{32}$ different values. A 64-bit
machine has $2^{64}$ different values and so on. For the case where
there is no sign, it is obvious to map the representation into $0,
\dotsc, 2^{32}-1$ and do calculation modulo $2^{32}$. Coincidentially,
this is how most modern computers work anyway.

JANUS choses this representation as well and models a 32-bit machine
in the process. For \januso{}, we would like to model the same thing
as full JANUS.

From the perspective of mathematics, 32-bit values means we are
working in the additive group $\mathbb{Z}/2^{32}\mathbb{Z}$. We could
have chosen other groups as the base for our numbers in JANUS. In
fact, any cyclic additive group will do. If we choose the number to
mod by as a prime number, mulitplication will have an inverse as
well\footnote{We would have a mathematical field, $\mathcal{F}_p$},
but the idea is not idiomatic with computers: they tend to use powers
of $2$ for their word size.

Let us recap exactly what is meant by an commutative group:
\begin{defn}
  An commutative group is a set $G$ with a binary operator $+\colon G
  \times G \to G$ such that it fulfills the following properties:
  \begin{itemize}
  \item For $x,y,z \in G$ we have $(x + y) + z = x + (y + z)$.
  \item There exist an element $0 \in G$ such that for any element $x
    \in G$ we have $x + 0 = 0 + x = x$.
  \item For any element $x \in G$ there exist an \emph{inverse}
    element, notated $-x$, such that $x + -x = -x + x = 0$
  \item For elements $x, y \in G$ we have $x + y = y + x$
  \end{itemize}
  The first rule is called associativity; the second is the existence
  of a neutral element (neutral identity) and the third is the
  existence of an inverse element. The fourth rule defines that the
  binary operation is commutative.

  If the fourth rule is omitted, we talk about a group. This is the
  normal definition to which one adds commutativity to get a
  commutative group.
\end{defn}

One particular commutative group is well-known and a good candidate
for a number base for JANUS; the usual group of integers
$\mathbb{Z}$. Implicitly, this was the choice for \janusz{}. From a
formal viewpoint, this group is bliss to work with. On the other hand,
it means that the formal machine have arbitrarily sized integers,
which may be a bit unrealistic. Hence, we chose numbers modulo
$2^{32}$ as we can easily formalize 64-bit numbers as well if needed
by changing a constant. We will call these integers $W^{32}$.

\subsection{\coq{} Implementation}

The Coq implementation of 32-bit numbers owes everything to Xavier
Leroy et al\cite{Leroy-Compcert-Coq}. The work is taken from the
\textsc{CompCert} project, which introduces them in the course of
formalizing a complete C to PowerPC compiler. The reason for taking
the work rather than building it from scratch is due to the sheer
complexity of getting it right. The formalization uses standard
algebraic properties to systematically build the necessary knowledge
and properties in \coq{}.

We did some simplification to the full specification as we only needed
parts of it. We also added a couple of new properties to the
development which is needed for proving \januso{}.

We can't resist the temptation to describe the implementation however. A
32-bit number is a pair: An integer $i$ and a proof that $0 \leq i <
2^{32}$. The main theorem is:
\begin{lem}
  Let $x$ be in integer. For any such $x$ we have
  \begin{equation*}
    0 \leq (x \;mod\; 2^{32}) < 2^{32}
  \end{equation*}
\end{lem}
\begin{proof}
  From our Coq Development:
\begin{verbatim}
Lemma mod_in_range:
  forall x, 0 <= Zmod x modulus < modulus.
  intros.
  exact (Z_mod_lt x modulus (two_power_nat_pos word_size)).
Qed.
\end{verbatim}
  Which states that it is exactly given by the common result
  \texttt{Z\_mod\_lt} from the Coq standard library.
\end{proof}
The theorem allows us to take any integer and represent it as a 32-bit
number as long as we take the integer $\mod 2^{32}$. Thus we have a
straightforward embed/project pair of functions between normal
integers and 32 bit integers.

For example, here is the definition of addition:
\begin{verbatim}
Definition add (x y: w32) : w32 :=
  repr (unsigned x + unsigned y).
\end{verbatim}
The function \texttt{unsigned} projects $x$ and $y$ to integers. Then
we integer-add the numbers and re-embed them with the \texttt{repr}
function. Note that the embedding is doing the correct thing for
cyclic additive groups: it wraps around.

Bitwise operations in the representation is implemented by creating
indicator functions from numbers up to a certain size in bits. Ie a
number is converted into a function $f \colon \mathbb{Z} \to \{0,
1\}$, where $f(n) = 1$ indicates $n$-th bit is set and $f(n) = 0$
that it is not. Straightforward combinations and embeddings of these
functions builds up the bitwise functions.

The properties of the numbers are simply a direct (rather impressive)
work of standard algebra. The theorems can be found in any math-book
on algebra, for instance \cite{thorup:algebra}. First, we have
\begin{defn}
  Two integers $x, y$ are in the same congruence class modulo $m$ iff
  there exists an integer $k$ such that
  \begin{equation*}
    x = k \cdot m + y
  \end{equation*}
\end{defn}
With this definition Xavier and co. then proceeds to build up a
complete library of known algebraic properties for congruence classes
like the one above. As an example we present the addition of
congruence classes:
\begin{verbatim}
  Lemma eqmod_add:
    forall a b c d, eqmod a b -> eqmod c d
      -> eqmod (a + c) (b + d).
  Proof.
    intros a b c d [k1 EQ1] [k2 EQ2]; red.
    subst a; subst c. exists (k1 + k2). ring.
  Qed.
\end{verbatim}
If $a, b$ is a congruence class pair and $c, d$ also is, then the pair
$a+c, b+d$ is an equivalence pair (for a fixed modulo). The proof proceeds by the
well-known method that there must exist a $k_1$ by the first pair such
that $a = k_1 \cdot m + b$. Likewise $c = k_2 \cdot m + d$. Adding
these 2 equations and rearranging yields $a + c = (k_1 + k_2) \cdot m
 + (b + d)$ which is the form one wants. Most of the proofs reflect
usual mathematical methods like this.

\paragraph{Additions to the CompCert integers}

We added some additional properties to the \textsc{CompCert}
implementation of 32-bit integers. All the proofs of these are written
in \coq{} by the use of already given properties given by Leroy
et.al. We extend their definitions by:
\command{xor[\;xor\;]}
\begin{lem}
  For any $x \in W^{32}$ we have $x \<xor> x = 0$
\end{lem}
\begin{lem}
  For 32-bit integers $x,y,z$, additions can be discharged on the
  left: $x + y = x + z \implies y = z$
\end{lem}
\begin{lem}
  For 32-bit integers $x, y, z$, subtracition can be discharged on the
  right: $x - z = y - z \implies x = y$
\end{lem}
\begin{lem}
  For 32-bit integers $x, y, z$, xor can be discharged on left:
  $x \<xor> z = y \<xor> z \implies x = y$.
\end{lem}
\subsection{Adapting \janusz{} to \januso{}}

We then alter \janusz{} to use 32-bit integers. First, we utilize the
functor on stores to get a store for 32-bit integers. Then, we
systematically change any mention of the type \texttt{Z} into the type
\texttt{w32}, the latter being the base type for 32-bit integers. This
in turn provokes changes of all arithmetic operators on \texttt{Z} to
their 32-bit counterparts.

After this, rather straightforward treatment, \januso{} now has 32-bit
integers as the base value.

\section{Augmenting expressions}

To the expressions of \janusz{}, we make the following additions:
\begin{align*}
  e ::= & \dotsc \bor e / e \bor e \mod e \bor e = e \bor e \neq e \\
        \bor & e \land e \bor e \lor e \bor e < e
\end{align*}

The additions to the expression language are as in \cite{glueck+2008};
we add aritmethic operators of division and modulo to 32-bit
numbers. We add a couple of relational operators, namely equality,
non-equality, logical and, logical or and a less-than operator.

With these additions, we almost have all constructs from the full
JANUS language. It is my belief that these can be added without any
problems at all. Specifically, our 32-bit integer representation
provisions for bitwise operators.

The semantics are given by a denotational semantics in the spirit of
the denotational semantics in \ref{exp:denot-semantics}. We omit them
here for brevity and refer the interested reader to the \coq{}
development.

\fixme{Describe encoding in \coq{}}

\section{Augmenting statements}

To the statement language we add the rules of \texttt{call},
\texttt{uncall} and xor ($\hat{=}$). 
\begin{equation*}
  s ::= \dotsc \bor \<call> p \bor \<uncall> p \bor x \; \hat{=} \; e
\end{equation*}
where $p$ designates a procedure. We will see how we encode this
procedure shortly.

The semantical judgement forms of statements are altered to a form
$\rho |= \angel{\sigma, s} -> \sigma'$, where $\rho$ is a map from $\ZZ \to
Stm$. That is, we represent a procedure by an integer and its value in
the set of procedure definitions is a statement. We note that all
existing rules in the operational semantics just has to pass $\rho$ on
via congruence. Hence, we do not bring these rules here. For the new
rules, there is an interesting development. The rule of calling a
procedure is
\begin{equation*}
  \inference[Call]{\rho |= \angel{\sigma, \rho(p)} -> \sigma'}
  {\rho |= \angel{\sigma, \<call> p} -> \sigma'}
\end{equation*}
Informally, the rule states that a procedure call looks up the
procedure in the context and then executes the procedure. Uncalling a
procedure is defined as
\begin{equation*}
  \inference[Uncall]{\rho |= \angel{\sigma', \rho(p)} -> \sigma}
  {\rho |= \angel{\sigma, \<uncall> p} -> \sigma'}
\end{equation*}
In this rule, we utilize the full power of operational-semantics given
as a prolog-style rule. Rather than relying on the inversion operator
we will define shortly, we alter the order procedure will run. This is
the main reason we choose an operational semantics rather than a
denotational one as the latter would have a harder time representing
this specific rule.

For the xor-operation, we use a rule which reflects the increment and
decrement operator:
\begin{equation*}
  \inference[Xor]{(\sigma \setminus x) |- e => k \quad \sigma(x) =
    \lift{k'} \quad k \oplus_{32} k' = n}{\angel{\sigma, x \;\hat{=}\; e} -> \sigma[x \mapsto n]}
\end{equation*}
where $\oplus_{32}$ is the xor-operation.

\fixme{Describe encoding in \coq{}}

\section{Determinism properties of \januso{}}

Converting the proofs from \janusz{} to \januso{} has one main
complication: we went from $\ZZ$ to $W^{32}$. When \coq{} works with
integers, and we only used Presburger
arithmetic\cite{cooper+1972:theorem-prooving} we can solve such terms
by the use of the ``omega'' tactic in \coq{}. For our 32-bit integers,
no such thing is present however, so we need to do the proofs by hand.

The detail at which one has to carry out such proofs are almost
exruciatingly precise. We must, as an example, explicitly use the
commutative rule when we want to rewrite $x +_{32} y$ into $y +_{32}
x$.

Irregardless, we have forward and backward determinism. However, the
formalization in \coq{} is slightly different to work around a problem
with mutually defined theorems. This means we cannot utilize the
theorem in further development as it is given in the development.
\begin{thm}
\label{thm:j1-fwd-det}
  \januso{} is forward deterministic, ie:
  \begin{equation*}
    \forall \sigma, \sigma', \sigma'' \in \Sigma, s \in Stm \colon
    \rho |= \angel{\sigma, s} -> \sigma' \implies \rho |= \angel{\sigma, s} -> \sigma'' \implies \sigma' = \sigma''
  \end{equation*}
\end{thm}
\begin{proof}
  In \coq{}
\end{proof}
and
\begin{thm}
\label{thm:j1-bwd-det}
  \januso{} is backward deterministic, ie:
  \begin{equation*}
    \forall \sigma, \sigma', \sigma'' \in \Sigma, s \in Stm \colon
    \rho |= \angel{\sigma', s} -> \sigma \implies \rho |= \angel{\sigma'', s} -> \sigma \implies \sigma' = \sigma''
  \end{equation*}
\end{thm}
\begin{proof}
  Formalized in \coq{}
\end{proof}

\section{Inversion and its properties}

Inverting \januso{} works exactly like inverting \janusz{}. We only
need to add a couple of new rules for $\<call>$ and $\<uncall>$ and
the $\hat{=}$ operation. The two call types are each others inverses
and the $\<xor>$ operation is self inverse:
\begin{align*}
    \mathcal{I}(x \;\hat{=}\; e)& = x \;\hat{=}\; e\\
    \mathcal{I}(\<call> p)& = \<uncall> p\\
    \mathcal{I}(\<uncall> p)& = \<call> p
\end{align*}

With these rules, we can again prove the inversion sound by the
following theorem:
\begin{thm}
  Let $\rho$ designate an arbitrary \januso{} program, $s$ an
  arbitrary \januso{} statement and let $\sigma$ and $\sigma'$ be
  arbitrary stores. The following is then true:
  \begin{equation*}
    \rho |= \angel{\sigma, s} -> \sigma' \iff \rho |= \angel{\sigma', \mathcal{I}(s)} -> \sigma
  \end{equation*}
\end{thm}



%%% Local Variables:
%%% mode: latex
%%% TeX-master: "master"
%%% End:
