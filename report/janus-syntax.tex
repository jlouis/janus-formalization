\chapter{JANUS Syntax}

The following section describes the syntax of the JANUS language
dialect we are working with. Rather than a top-down approach, we will
introduce the components of the syntax when needed.

\paragraph{Expressions}
 
In JANUS, we have expressions. They have the following grammar:
\begin{align*}
  e ::& = w^{32} \;\vert\; x \\
        & \;\vert\; e_1 + e_2 \;\vert\; e_1 - e_2 \;\vert\; e_1 *
          e_2 \;\vert\; e_1 / e_2 \;\vert\; e_1 \;\%\; e_2 \;\vert\; e_1 /\!* e_2 \\
        & \;\vert\; e_1 \otimes e_2 \;\vert\; e_1 \;\&\&\; e_2 \;\vert\;
        e_1 \;||\; e_2 \\
        & \;\vert\; e_1 = e_2 \;\vert\; e_1 \neq e_2 \;\vert\; e_1 \;\&\;
        e_2 \;\vert\; e_1 \;\|\; e_2 \;\vert\; e_1 < e_2 \;\vert\; e_1
        > e_2 \;\vert\; e_1 \leq e_2 \;\vert\;e_1 \geq e_2 \;\vert\;
\end{align*}
The value $w^{32}$ is a 32-bit unsigned integer. All the familiar
arithmetic expressions are arithmetic on 32-bit integers as
expected. The operations $\%$ is modulo as in the C programming
language. \fixme{describe fracprod}

There are three bitwise operators in the language. These operate
bitwise on their operands, producing the bitwise result. In order they
are the Xor operation, the And operation and finally the Or
operation.

There are a number of relational operators in the language. These will
return the value $1$ if true and $0$ if false, again in correspondance
with C. For the logical And-operation and the logical Or-operation,
falsehood of an operand is $0$, everything else is true as one
expects.

\paragraph{Statements}
The statement language of JANUS has the following syntax:
\begin{align*}
  s ::&= ... \\
      &\;\vert\; ...
\end{align*}

%%% Local Variables: 
%%% mode: latex
%%% TeX-master: "master"
%%% End: 
