\newcommand{\janusz}{$\mathrm{JANUS}_0$}
\newcommand{\coq}{\textsc{Coq}}
\newcommand{\ZZ}{\mathbb{Z}}
\chapter{\janusz{}}

The \janusz{} programming language is a subset of the full JANUS
language from \cite{glueck2007}. It not a Turing Complete language and
contains only simple linear constructs. However, it contains enough to
provide a vehicle for explaining the basics of a formalization. We aim
to provide a big step operational semantics for the language, suitable
for encoding into the logical framework \coq{}.

Describing a language requires us to define the basic object that is
manipulated. There are two such objects in \janusz{}: integers and
stores.

The integers are the mathematical integers, ie drawn from $\ZZ$. In
full JANUS the primary object are 32-bit integers but in our
simplified version, we just use usual integers.

\newcommand{\lift}[1]{\lfloor #1 \rfloor}
\begin{defn}
  For any set, $S$ we define its domain-theoretic \emph{lift} to be
  $S_{\perp} = S \cup \{\perp\} $ for a special value $\perp$ called
  ``bottom''. Values $s \in S$ are notated as $\lift{s}$ which is
  called the ``lift''.
\end{defn}
Information-theoretically, the bottom values represents ``no
information'' whereas the lift of a value represents that value. This
is akin to the well-known ML datatype ``option''.

The stores, notated as $\rho, \rho', \dotsc$ are functions from
integers to lifted integers: $\rho \colon \ZZ \to \ZZ_{\perp}$. The
domain of integers are called \emph{locations}. We usually give them
names $x, y, z, \dotsc$ and assume each variable is associated with a
specific integer for the duration of the program. The co-domain
defines the contents of the current location in the store. If no value
is associated with the location $x$ then $x \mapsto \perp$ and
otherwise $x \mapsto \lift{i}$ for some $i \in \ZZ$. This process is
used as lookup.

Stores are altered under the course of running a program. In our
formalization this amounts to changing the function
$\rho$. The notation $\rho[x \mapsto k]$ means ``update the contents
of location $x$ with the value $k$''. The mathematical formal meaning
is:
\begin{equation*}
  \rho[x \mapsto k](z) = \begin{cases}
    \lift{k} & \quad x = z\\
    \rho(z)  & \quad \text{otherwise}
  \end{cases}
\end{equation*}

The empty store $\epsilon$ maps everything to $\perp$, ie. for all
locations $l \in \ZZ$ we have $\epsilon(l) = \perp$. We will use the
empty store as the inital store.

\subsection{Coq encoding}

In \coq{} there are already integers built-in, ready for use in our
formaliztion. This means we can focus entirely on providing an
implementation of stores.

A Store maps 

\section{Expressions in \janusz{}}

The \janusz{} language has a very simplified expression language
compared to full JANUS. In this language there are 5 expression
constructs: integer constants, store referencing, addition,
subtraction and multiplication.

The syntax of expressions $e$ is the following in BNF notation:
\newcommand{\bor}{\; | \;}
\begin{equation*}
  e ::= n \bor \mathtt{x} \bor e + e \bor e - e \bor e * e
\end{equation*}
The judgement forms are $\rho |- e => z$ stating that under the
assumption of a store $\rho$ the expression $e$ evaluates to the
integer $z$. The inference rules for this system is straightforward:
\begin{gather*}
  \inference[Const]{}{\rho |- n => n} \quad \inference[Var]{\rho(\mathtt{x}) =
    \lift{k}}{\rho |- \mathtt{x} => k} \\
  \inference[Add]{\rho |- e_1 => n_1 \quad \rho |- e_2 => n_2 \quad
    n_1 + n_2 = n}{\rho |- e_1 + e_2 => n} \\
  \inference[Sub]{\rho |- e_1 => n_1 \quad \rho |- e_2 => n_2 \quad
    n_1 - n_2 = n}{\rho |- e_1 - e_2 => n} \\
  \inference[Mul]{\rho |- e_1 => n_1 \quad \rho |- e_2 => n_2 \quad
    n_1 * n_2 = n}{\rho |- e_1 * e_2 => n}
\end{gather*}

\section{Statements in \janusz{}}

The statement syntax of \janusz{} defines a small, purely linear
subset of the full JANUS language. There are 4 constructs given via
the following notation in BNF:
\reservestyle{\command}{\mathbf}
\command{if[\;if\;],then[\;then\;],else[\;else\;],fi[\;fi\;]}
\begin{gather*}
  s ::= x +\!\!= e \bor x -\!\!= e \bor s; s
  \\ \bor \<if> e \<then> s \<else> s \<fi> e
\end{gather*}
We will interleave the description of each of these syntactical
elements with the inference rules for them. The judgement form for
execution of statements is $\rho |- s -> \rho'$ which designates that
under the assumption of a store $\rho$, execution of statement $s$
will yield the altered store $\rho'$.

The first operation is the increment operation of an element in the
store. This operation, written as $x +\!\!= e$ will evaluate the
expression $e$ to a number $k$ and then add this amount to the
location in the store to which $x$ points:
\begin{equation*}
  \inference[Inc0]{\rho |- e => k \quad \rho(x) = \lift{k'} \quad k +
    k' = n}{\rho |- x +\!\!= e -> \rho[x \mapsto n]}
\end{equation*}
In JANUS it is a requirement that the variable $x$ must not occur in
the expression $e$. The same requiremnt is present in \janusz{} and
has to do with the invertibility of such statements. There is,
however, an alternative semantics not present in the current
literature which directly encodes the requirement in the inference
rule.

Recall we defined an ability to ``hide'' certain variables in our
store. We can utilize this by hiding $x$ in the expression evaluation:
\begin{equation*}
  \inference[Inc]{(\rho \setminus x) |- e => k \quad \rho(x) =
    \lift{k'} \quad k + k' = n}{\rho |- x +\!\!= e -> \rho[x \mapsto n]}
\end{equation*}
Now, because expression evaluation of the ``Var'' case requires a
lifted value it is now impossible to construct an inference tree where
the expression $e$ refers to the value $x$. We have effectively
encoded the informal requirement into a formal one. This method,
correctness by construction, simplifies proof formalization: had we
chosen a predicate judgement for an non-occurring variable $x$, then
we would have to provide a proof with this added structure.

For subtraction, the definition is similar:
\begin{equation*}
  \inference[Sub]{(\rho \setminus x) |- e => k \quad \rho(x) =
    \lift{k'} \quad k - k' = n}{\rho |- x -\!\!= e -> \rho[x \mapsto n]}
\end{equation*}

The next rule is for sequencing operations. In the statement $s_1;
s_2$ one first executes $s_1$ and then feeds the resulting store into
the execution of $s_2$:
\begin{equation*}
  \inference[Seq]{\rho |- s_1 -> \rho'' \quad \rho'' |- s_2 -> \rho'}
  {\rho |- s_1; s_2 -> \rho'}
\end{equation*}

Finally, there is the rule for the branching instruction. In \janusz{}
the value $0$ is ``false'' and any value different from $0$ is the
``true'' value. This yields two rules, one for each case. The first
rule is for the ``false'' case:
\begin{equation*}
  \inference[If-false]{\rho |- e_1 => 0 \quad \rho |- s_2 -> \rho'
    \quad \rho' |- e_2 => 0}{\<if> e_1 \<then> s_1 \<else> s_2 \<fi>
    e_2 -> \rho'}
\end{equation*}
This rule states that if the $e_1$ \emph{test} evaluates to a false
value, then the ``else''-branch is executed. Finally the
\emph{assertion} $e_2$ must be false as well.

The true case is similar, the difference being extra (mathematical)
requirements on what the expressions evaluates to:
\begin{equation*}
  \inference[If-true]{\rho |- e_1 => k \quad k \neq 0 \quad \rho |- s_1 -> \rho'
    \quad \rho' |- e_2 => k' \quad k' \neq 0}{\<if> e_1 \<then> s_1 \<else> s_2 \<fi>
    e_2 -> \rho'}
\end{equation*}

The need for assertions has to do with reversibility, as we shall see
later on.

%%% Local Variables: 
%%% mode: latex
%%% TeX-master: "master"
%%% End: 
