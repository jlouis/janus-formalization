\documentclass[a4paper, oneside, 10pt]{memoir}
\chapterstyle{culver}
\usepackage{fixme}
\usepackage[osf,sc]{mathpazo}
\linespread{1.05}
%\usepackage[utopia]{mathdesign}
\input{../../Preamble/preamble_memoir.tex}

\usepackage{semantic}
\author{Jesper Louis
  Andersen\\jesper.louis.andersen@gmail.com\\140280-2029}
\title{A Formalization of JANUS}
\date{\today}

\renewcommand*{\titleM}{\begingroup% Misericords, T&H p 153
  \drop = 0.08\textheight
  \centering
  {\Huge\bfseries JANUS}\\[\baselineskip]
  {\scshape a mechanized verification}\\[\baselineskip]
  {\scshape by}\\[\baselineskip]
  {\large\scshape Jesper Louis Andersen\\jesper.louis.andersen@gmail.com}\par
  \endgroup}

\bibliographystyle{amsalpha}
\begin{document}
\titleM

\paragraph{Project Description}

The JANUS programming language is a reversible programming language,
suitable for research in reversible computation. This 15 ECTS project
has the goal of providing a full, machine-verified formal semantics
for the JANUS programming language.

We plan to start with the formal semantics present in the literature
(\cite{glueck2007}, \cite{glueck2008}) and then formalize the
semantics and a couple of key proofs in a proof assistant. The goal is
not to show the existing proofs wrong, but rather to sharpen the
details of the proofs and alter the semantics to make proof simpler.

In the first phase of the project, we will work with a simplified
version of the language called $\mathrm{JANUS}_0$. This version will
let us introduce the concepts in a pedagogical way and build up
towards a full formalization -- but it will probably not be a Turing
Complete language.

Next, we will introduce $\mathrm{JANUS_1}$ which is a larger subset of
the full JANUS language. Finally, we will use the bootstrapping
sequence to provide a full or nearly full formalization of the
complete JANUS language.

The framework for formalization and mechanical, computer aided
verification, is not decided at project start, though the contenders
are the ``Twelf logical framework'' (\cite{harper+07:mechanizing},
\cite{hhp93lf}, \cite{twelf}) and the ``Coq proof assistant'' (\cite{coqdev}).
We intend to deliver:
\begin{itemize}
\item A full encoding of the formal semantics from the literature.
\item An encoding and verification of key proofs.
\item A report describing the methods used.
\item Eventual alterations of the semantics in order to simplify the
  proofs.
\end{itemize}

\subparagraph{Outcome}
\label{sec:consequence}

It might very well be that the academic contribution will come from
the reshaping of the semantics. In the case that the proofs and
encoding are already perfect, the academic contribution will then be
the formalization work.

\paragraph{Learning goals}

Writing a graduate project is as much a learning situation as it is
academic work. Hence, we have several learning goals associated with
the project.

At the end of the project, I will have learned:
\begin{itemize}
\item How to write down formal semantics for a medium sized
  project. While I have done smaller semantics as part of exercises in
  courses, this endeavour is much more involved.
\item How to verify proof correctness by the use of formal
  frameworks. The verification will make sure that only sound valid
  logical rules are being used so any kind of logical mistake is
  corrected.
\item How to \emph{program} proofs in the logical framework Coq for
  larger developments. This is largely a skill that one can improve
  over time, much akin to programming.
\end{itemize}
In addition I will have gained knowledge about:
\begin{itemize}
\item The details of the JANUS language and reversible programming in
  general. A formal development will force us to acquire intimate
  knowledge of the language in particular.
\item The Coq formal framework. While I have used it before, it was
  rather superficial. This work will force me to gain much more
  knowledge about its intrisics.
\end{itemize}

These learning goals will enable further studies in formal frameworks
as well as reversible programming.

\bibliography{biblio} 
% What is the project about?
% What am I going to learn?
% What is the ECTS valuation
\end{document}
%%% Local Variables: 
%%% mode: latex
%%% TeX-master: t
%%% End: 
