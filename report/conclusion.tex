\chapter{Conclusion}

\fixme{This conclusion needs some updating!}
In this project we have:
\begin{itemize}
\item Formalized most of JANUS and its main properties, if we
  disregard loops. The formalization is carried out in the \coq{}
  proof assistant, amounts to 1500 lines of vernacular code. This must
  be considered an intermediate-size formalization.
\item Fncovered a problem in the formalization of loops with
  the existing semantics for JANUS.
\item Shown that the ability to do \emph{hiding} in stores
  yields a pretty solution to the problem ``$x$ must not occur in the
  expression on the right-hand side''. Thus, we have contributed more
  formality to the process of writing formalizations of reversible
  languages. Many reversible languages with stores will gain benefit
  from the idea of store hiding.
\end{itemize}

Furthermore we gained important insight in the use of the \coq{}
framework for formalization of computer programs in general. This
insight will benefit further work with logical frameworks or proof
assistants as many concepts are interchangeable between them. We also
provide a foundation to build on for further formalization of a
JANUS-like language as it is far easier to change a formalization than
it is to provide a new one from scratch.

%%% Local Variables: 
%%% mode: latex
%%% TeX-master: "master"
%%% End: 
