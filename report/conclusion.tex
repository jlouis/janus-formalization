\chapter{Conclusion}

% [Conclusions are very important. Do not expect that the reader remembers everything you told him/her.
% Having stated the definitions, you can now be more specific that  in the introduction]
% * Overview what this work was about.
% * Main results and contributions
% * Comments on importance or
% * Tips for practical use [how your results or experience can help someone in practice or
%     another researcher to use your simulator or avoid pitfalls]
% * Future work. Reinforce the importance of work, but avoid giving out your ideas].

In this project we have studied the machine verifiable formalization
of the JANUS programming language. We gradually built up the language
in several steps towards the full JANUS language. Our accomplishments are:
\begin{itemize}
\item Formalization of large parts of the JANUS language. The
  formalization is machine-verified in the \coq{} proof assistant, and
  amounts to 1500 lines of vernacular code (of own production).
\item Uncovering of a problem with the backwards determinism proof. It
  is clear a simple proof is not possible. One needs a more complex
  proof or an altered semantics. For the latter, it must still
  encapsulate the ideology of the JANUS language.
\item Contribution of the \emph{hiding}-concept in stores for encoding
  of the informal notion ``$x$ must not occur in expression $e$''. Many
  reversible languages with stores will benefit from this idea as it
  is directly applicable.
\item Foundations for formalization of reversible languages. It is far
  easier to amend an almost-full JANUS formalization than it is to
  begin from scratch. As such the work can be used for further studies.
\end{itemize}

In addition, we learned important lessons on using proof frameworks,
in particular the framework of \coq{}. And we learned a lot about
encoding syntax, semantics and proofs. This knowledge is useful with
other proof frameworks.

%%% Local Variables: 
%%% mode: latex
%%% TeX-master: "master"
%%% End: 
