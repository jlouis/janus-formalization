\chapter{Full JANUS}

The full JANUS language adds 2 things to \januso{}: arrays and
loops. Taking loops first, we augment the syntax of statements with
\command{loop[\;loop\;],from[\;from\;],do[\;do\;],until[\;until\;]}
\begin{equation*}
  s ::= \dotsc \bor \<from> e \<do> s \<loop> s \<until> e
\end{equation*}
The idea of the loop is probably best described by its formal
semantics. To make this work, a new judgement relation for loops is
introduced with notation $\rho |=_{loop} \angel{\sigma, (e_1, s_1,
  s_2, e_2)} -> \sigma'$. This states the evaluation of the loop
identified by the quadruple $(e_1, s_1, s_2, e_2)$ under the function
definition $\rho$ and the store $\sigma$ will yield an updated store
$\sigma'$.

We then alter the judgement rules of statements by the following
looping rule:
\begin{equation*}
  \inference[Loop]{\sigma |- e_1 => k \quad k \neq 0 \\
    \rho |= \angel{\sigma, s_1} -> \sigma'' \quad \rho
    |=_{loop} \angel{\sigma'', (e_1, s_1, s_2, e_2)} -> \sigma'}
  {\rho |= \angel{\sigma, \<from> e \<do> s \<loop> s \<until> e} -> \sigma'}
\end{equation*}
This rule will only be valid if $e_1$ evaluates to a true value. Then
it runs $s_1$ and enters the loop proper.

Simultaneously with the definition of statements, we have the 2 rules
on the $|=_{loop}$ judgement form, taken from \cite{glueck+2008}. The
first one exits the loop then the ``until'' part evaluates to true:
\begin{equation*}
  \inference[LpT]{\sigma |- e_2 => k \quad k \neq 0}
  {\rho |=_{loop} \angel{\sigma, (e_1, s_1, s_2, e_2)} -> \sigma}
\end{equation*}
Finally, if the ``until'' part evaluates to false, we take another
turn round the loop. The following rule captures this:
\begin{equation*}
  \inference[LpF]{\sigma |- e_2 => 0 \\
                  \rho |= \angel{\sigma, s_2} -> \sigma''\\
                  \sigma |- e_1 => 0 \\
                  \rho |= \angel{\sigma'', s_1} -> \sigma'''\\
                  \rho |=_{loop} \angel{\sigma''', (e_1, s_1, s_2,
                    e_2)} -> \sigma'}
  {\rho |=_{loop} \angel{\sigma, (e_1, s_1, s_2, e_2)} -> \sigma'}
\end{equation*}

\paragraph{Arrays}

Adding arrays to \januso{} is not something I've carried out in
\coq{}. It is left as further work. The hypothesis is that they would
be straightforward to add. An array is a map from $W^{32}$ to
$W^{32}_{\perp}$. We could then construct a representation as a
(higher order) function $\ZZ \to W^{32}_{\perp} \to W^{32}_{\perp}$ in
\coq{}. The proof would then reflect the methodology we used for
variables.

If course, we could need to alter expressions to be able to do
reference cells in arrays. We would also need to formalize the
semantics in the case we go out-of-bounds on the array. All this is
not hard, but would impact a rather wide of the formalization already
done.

\section{Properties of full JANUS}
\label{sec:prop-full-janus}

I initially set out with the prospect of proving forwards and
backwards determinism for the full JANUS language given in
\cite{glueck+2007} which has a different semantics for
loops. Unfortunately, the proof relies on a property of there only be
one specific way a given tree can be constructed. This property is
rather hard to formalize. Hence, I changed to the semantics given
above from \cite{glueck+2008}.

The language can be proven to be forwards deterministic (under the
assumption that it is backwards deterministic). But I failed to find a
machine-verifiable proof that it is backwards deterministic. In the
following sections I will attempt to describe what the problem is.

We are trying to prove the following statement:
\begin{multline*}
  \forall \rho, \sigma, \sigma', \sigma'', s . \\
  \rho |= \angel{\sigma', s} -> \sigma \implies \rho |=
  \angel{\sigma'', s} -> \sigma \implies \sigma' = \sigma''
\end{multline*}

The proof proceeds by induction on the first dependent hypothesis,
that is $\rho |= \angel{\sigma', s} -> \sigma$. This splits up into
cases for $|=$ and simultaneously for $|=_{loop}$. To get \coq{} to
understand this simultaneous induction principle, the \texttt{Scheme}
command must be used.

All cases for the $|=$ judgement are easily discharged. The only new
rule we have added is the one of the form:
\begin{equation*}
  \inference[Loop]{\sigma |- e_1 => k \quad k \neq 0 \\
    \rho |= \angel{\sigma, s_1} -> \sigma'' \quad \rho
    |=_{loop} \angel{\sigma'', (e_1, s_1, s_2, e_2)} -> \sigma'}
  {\rho |= \angel{\sigma, \<from> e \<do> s \<loop> s \<until> e} -> \sigma'}
\end{equation*}
Now looking at the second dependent hypothesis, namely $\rho |=
\angel{\sigma', s} -> \sigma$. By inversion, it can only be an
instance of the rule $Loop$ so we can use the induction hypothesis
generated on $\rho |=_{loop} \angel{\sigma'', (e_1, s_1, s_2, e_2)} ->
\sigma$ to conclude $\sigma' = \sigma''$. Case done.

But the cases for $|=_{loop}$ are not easily discharged. Suppose we
have the case $\rho |=_{loop} \angel{\sigma',
  (e_1, s_1, s_2, e_2)} -> \sigma$. By inversion, we generate two
possible rules that could match. One is:
\begin{equation*}
  \inference[LpT]{\sigma |- e_2 => k \quad k \neq 0}
  {\rho |=_{loop} \angel{\sigma, (e_1, s_1, s_2, e_2)} -> \sigma}
\end{equation*}

But by inversion, the 2nd inductive hypothesis $\rho |=_{loop} \angel{\sigma'',
  (e_1, s_1, s_2, e_2)} -> \sigma$ could have generated the $LpF$
case:
\begin{equation*}
  \inference[LpF]{\sigma'' |- e_2 => 0 \\
                  \rho |= \angel{\sigma'', s_2} -> \sigma''''\\
                  \sigma'''' |- e_1 => 0 \\
                  \rho |= \angel{\sigma'''', s_1} -> \sigma'''\\
                  \rho |=_{loop} \angel{\sigma''', (e_1, s_1, s_2,
                    e_2)} -> \sigma}
  {\rho |=_{loop} \angel{\sigma'', (e_1, s_1, s_2, e_2)} -> \sigma}
\end{equation*}

Normally when this happens in if-then-else or in the forward direction
of loops, we \emph{discriminate} the cases. We find something which is
true in one case and false in the other. This makes the combination
absurd so we can dismiss the possibility of the case. In the forward
direction we would have $\sigma |- e_2 => 0$ and at the same time
$\sigma |- e_2 => k \land k \neq 0$. But in the backwards direction
the discrimination utterly fails.

I then tried to look at using an argument on the height of the two
cases, but it will not work as expected: the 1st and 2nd dependent
hypotheses are different at their ``base''. In any case, there is no
``easy'' proof possible and there is need for some ingeinuity to
complete the proof.

One should \emph{not} be let down but this result however. It might be
there is another proof or it might be that we need to change the
semantics of loops to something equivalent that is easier to
prove. The loop currently defined relies on the fact that an
expression evaluation will be true only on the first and the last
iteration of loop. $e_1$ will be true upon entering the loop and $e_2$
will be true upon leaving the loop. Running the program backwards
reverses the role of $e_1$ and $e_2$. Interestingly, we define the
semantics in a way such that the evaluation of $e_1$ (in the forwards
direction) occurs in the $|=$ judgement whereas the evaluation of
$e_2$ occurs in the $|=_{loop}$ judgement. The problem with this split
is that we do not know when the loop began executing. This information
is lost on us when we process inductive cases for $|=_{loop}$. An
alternative semantics ought to correct this problem, perhaps by
adjusting the loop semantics from \cite{glueck+2007}

...

%%% Local Variables: 
%%% mode: latex
%%% TeX-master: "master"
%%% End: 
