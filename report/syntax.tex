\chapter{Syntax driven semantics}

The crucial insight in this formalization was gained by applying an
idea from the logic LF which is used in Twelf. In LF, the dependently
typed lambda calculus is restricted in a specific way. To control what
the valid terms are, an elaborate set of judgements are
constructed. The main point is that these inference rules drives the
valid forms: any constructed inference tree is valid. The feat is to
take great care when introducing the inference rules such that one
can't produce invalid rules, see \cite{harper+07:mechanizing}.

The same idea has been used in this JANUS formalization. In the
formalizations of \cite{glueck2007} and \cite{glueck2008}, certain rules
of the semantics are left in the text-paragraphs of the paper; notably
the requirement that a statement \texttt{x += e} must have the
variable \texttt{x} free in the body of \texttt{e}.

\fixme{Write this part}

%%% Local Variables: 
%%% mode: latex
%%% TeX-master: "master"
%%% End: 
