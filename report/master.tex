\documentclass[a4paper, oneside, 10pt, draft]{memoir}
\chapterstyle{culver}
\usepackage{fixme}
\usepackage{coqdoc}
\usepackage[english]{babel}
\usepackage[osf,sc]{mathpazo}
\linespread{1.05}
%\usepackage[utopia]{mathdesign}
\input{../../Preamble/preamble_memoir.tex}

\usepackage{semantic}
\author{Jesper Louis
  Andersen\\jesper.louis.andersen@gmail.com\\140280-2029}
\title{PIRC - Exam}
\date{\today}

\renewcommand*{\titleM}{\begingroup% Misericords, T&H p 153
  \drop = 0.08\textheight
  \centering
  {\Huge\bfseries Formalizing JANUS}\\[\baselineskip]
  {\scshape In a proof assistant}\\[\baselineskip]
  {\scshape by}\\[\baselineskip]
  {\large\scshape Jesper Louis Andersen\\jesper.louis.andersen@gmail.com}\par
  \endgroup}

\bibliographystyle{plain}
\hyphenation{in-di-ca-tor func-tions}

\begin{document}
\newcommand{\janusz}{$\mathrm{JANUS}_0$}
\newcommand{\januso}{$\mathrm{JANUS}_1$}
\newcommand{\lift}[1]{\lfloor #1 \rfloor}
\newcommand{\coq}{{\scshape Coq}}
\newcommand{\twelf}{{\scshape Twelf}}
\newcommand{\gallina}{{\scshape Gallina}}
\newcommand{\NN}{\mathbb{N}}
\newcommand{\ZZ}{\mathbb{Z}}
\titleM
\begin{abstract}
  In this project, we provide a state-of-the-art formalization on the
  reversible language JANUS. We obtain a
  new way of looking at reversible languages by combining it with
  machine-verified formal semantics. This has important implications
  in the theory and sets a new level for correctness of reversibility.

  We formalize the syntax, the semantics
  and we provide verification of main reversibility proofs in the
  proof assistant \coq{} amounting to over 1500 lines of vernacular
  definitions. The formalization covers large parts of JANUS and
  points to places that are impossible to prove with the currently
  given semantics. Furthermore we contribute improvements on the existing
  semantics by formalizing properties that were stated informally in
  the existing literature.

  To our knowledge, this is the first combination of mechanical
  verification with research in reversibility of programming
  languages.
\end{abstract}
\listoffixmes
\tableofcontents
\fixme{Ispell from here once everything is finished}
\chapter{Introduction}

The JANUS programming language were conceived as a student project in
1982 at Caltech by Chris Lutz and Howard Derby. It was first later
however, with the work in \cite{glueck+2007, glueck+2008} that
the language was given a modern formal treatment. In these papers, variants
of the language are established with operational semantics. The
results improves on earlier informal descriptions.

We study the JANUS programming language because it is an example of a
small reversible imperative language. These languages have the
interesting property that programs can be run backwards to obtain the
original input from the output of a computation. This study form the
base of programming languages for reversible CMOS technology and
quantum computing.

Usually when we add formality to a process, there is a need for
changing the original informal description. Formality enforces us to
optimize and make precise the definitions and results, which in turn
sharpens and refines our knowledge of a field. A better understanding
through formality makes it easier to gather new knowlegde in a field.

The current state-of-the-art with respect to formality is
machine-verifiable formalizations -- encoding our formal work in a way
the machine can understand. We then proceed to encode the meta-theory
of our formalization as well. Finally, we ask the machine to
systematically verify the correctness of our proofs.

The goal of this process is not to prove human beings wrong. It is
rarely the case that the checking process of ``the human grid
computer'' fails to notice errors in meta-theory. Rather, the goals
are those of simplicity, detailed understanding and domain knowledge:

When we explain a theorem to a computer, it is easiest to use simple
methods. It is more expensive to formalize a complex principle and use
it rather than taking more time and coming up with a simpler
proof. This forces us to use humility and simplicity in our work.

A proof explained to the computer can only draw on knowledge we
already told the computer. For any proof, we must be able to
explain any detail, as otherwise it will be impossible to
machine-verify.

Finally, we may hope to gain additional knowledge about the domain in
the formalization process. The precision required yield no place for
imprecision and cannot lure our intuition down the wrong path. Hence,
we often learn from the formalization process.

This DIKU graduate project establishes a full formalized semantics for
a variant of the JANUS language in the proof assistant \coq{}. The
main academic contribution of the paper is the use of syntax-driven
semantics for JANUS: Invalid programs are ruled out by having no possible
inference tree. The secondary contribution establishes the machine
verification of the main theorems of the JANUS variant.

To make the work digestible, it has been split in several subsets of
the JANUS language. First, we introduce some simplified versions,
\janusz{}, \januso{} -- before processing the complete language. It is
our hope that these subsets make it easier to understand the full
development.

The report also describes the toolbox needed for formalization in
\coq{} with the concepts introduced as they are needed.

Finally, this report is honest to the state-of-the-art: any result is
mechanically formalized and verified in \coq{}.

\subsection{Related work}

There are several examples of reversible programming languages. There
are reversible Turing machines \fixme{cite} and other reversible
functional languages \fixme{cite}. There are also many examples of
programming language formalizations: functional languages
\fixme{cite}, \fixme{cite}, lambda calculus \fixme{cite} and
imperative languages \fixme{cite}.

To our knowledge, nobody has tried to formalize reversible programming
languages using mechanical verification tools before this project. As
such, this project combines two otherwise unrelated fields: namely the
field of mechanical verification of programming language properties
and the field of reversible language theory. We hope this will benefit
both fields.

\subsection{Expectations}

We expect the reader to be well versed in the concepts of programming
language semantics and proofs of programming language meta-theory. We
further expect the reader to know the general idea of logical
frameworks, proof assistants and theorem provers. We won't describe
the concepts of \coq{} in much detail. We will just point to the
literature \fixme{cite}.

We also expect the reader to be somewhat familiar with the concept of
reversible languages. In particular we expect the reader to know what
it means for a language to be reversible in the informal sense.

\section{Choice of formalization framework}

One might want to know what decided the formal verification framework
to use. The author has used two frameworks in the past, \coq{} and \twelf{}
and it felt logical to use one of those in the formal verification.

One has to make the choice of framework early on and then stick to
it. It is exactly like choosing a programming language for a given
task.

\coq{} is a proof assistant using an (co-)inductive variant of the
Calculus of Constructions (CoC). It relies on the Curry-Howard
correspondence to get a link between program/proof and
types/theorems. Proof are carried out by giving tactics; hints about
what to do next. These tactics then build up a term which constitutes
the proof via the Curry-Howard isomorphism. Finally, a low level type
checker makes sure the proof is correct.

\twelf{} uses a dependently typed lambda calculus constrained to
certain canonical forms. This calculus is called LF. The syntax,
semantics, and proofs are then all encoded in this LF-language. For
proof validity, separate checker then verifies certain properties on
the correctness of the proof, see \cite{harper+05:how-to-believe,
  harper+07:mechanizing}.

\coq{} supports proof automation where parts of proofs are done
automatically by the computer. However, most of the time carrying out
proofs are spent on things other than the proof structure: a
considerable amount of time is spent on figuring out how to conduct
the proof. Another time consumer is feeding the necessary prior
knowledge to the proof system for the development. You can only carry
out 32-bit addition if the system knows what a 32-bit
number is, and what addition means. Properties of JANUS relies on
properties of 32-bit addition, so this must be fed to the framework as
well.

\twelf{} does not currently support automation of proof. On the other
hand, it provides proof \emph{adequacy}: there is an isomorphism
between the proof in \twelf{} and proof on paper. In other words, any
proof can be transcribed to a standard paper proof. If one is
interested in how the proof is carried out, \twelf{} has a more direct
approach. \coq{} does not provide a proof term that is easily
digestible by a human reader.

The JANUS development needs 32-bit numbers. These were available for
\coq{}, but not for \twelf{}. One could have used classic integers of
infinite size however. \coq{} further had support for arithmetic in
libraries, whereas in \twelf{}, we would have to encode this
ourselves. On the other hand, \twelf{} has a simpler system and excels
at working with lambda-style binders through
Higher-order-abstract-syntax (HOAS). For \twelf{}, it would be
beneficial to encode the functions via HOAS, but we can't utilize HOAS
for the store where it would have been even more beneficial. \twelf{}
also has support for several mutual induction schemes. These are also
available in \coq{}, but are considerably harder to use. From \coq{}
version 8.2, some syntactic sugar has been added to make its
use easier, but it has not seen much testing yet.

It was deemed that it would be best to have easy access to 32-bit
numbers and hence \coq{} were chosen. In hindsight certain mutual
induction proofs would probably have been easier to carry out in
\twelf{} and to rely on arbitrary sized integers would not have been
detrimental to the demonstration of the findings in this report. The
author would probably have used \twelf{} should he do the task again.

\section{Reversible computation}
\label{sec:revers-comp}

Suppose we are given a program $p$ and give it an input $x$ and it
yields an output $y$:
\begin{equation*}
  |[p|](x) = y
\end{equation*}
For some programs $p$ it will be possible to reconstruct $x$ from the
pair $(p, y)$.

To achieve this inversion property certain things must be
fulfilled. The program $p$ must at least be injective: Different
inputs must map the different outputs. Otherwise it is impossible to
reconstruct the input from the output in all cases. As an example,
take the Standard ML program
\begin{verbatim}
  fun even_odd x =
    x mod 2
\end{verbatim}
which will take any number $x$ and return either $0$ or $1$. All even
numbers will be assigned to $0$ and thus it is impossible to know
which even number was the input. The program is not injective and thus
not reversible.

\begin{defn}
  A program $p$ is called \emph{information preserving} if for
  $|[p|](x) = y$ it is possible to uniquely reconstruct $x$ from the
  pair $(p, y)$.
\end{defn}
Informally, no loss of information may occur in the computation.

\begin{thm}
  Any information preserving program is injective.
\end{thm}
\begin{proof}
  Assume the program is not injective. Then we have $x \neq y$ but
  $|[p|](x) = |[p|](y)$. but then we can't uniquely reconstruct $x$
  say since $y$ is equally likely.
\end{proof}
Injectivity is a necessary condition. The wording of information
preservation is carefully chosen such that there \emph{is} an input
yielding the given output. The output space is then the image of all
possible inputs, automatically making the statement surjective on the
co-domain.

For a reversible programming language, like JANUS, we obtain the
information preservation by means of an inversion property: Any
program $p$ can be inverted into a program $p^{-1}$ with
$|[p^{-1}|](y) = x$. Thus we can always reconstruct $x$ from the pair
$(p, y)$ by inverting $p$ and inputting $y$.

JANUS is reversible because it is possible to define a term-rewriting
system for turning a program into its inverse. We will see that it is
possible since each atomic instruction is invertible and each compound
instruction is invertible if its components are. This let us use an
inductive argument to show all programs reversible.

%%% Local Variables: 
%%% mode: latex
%%% TeX-master: "master"
%%% End: 

\chapter{Extensions to C(co-)iC}

The \coq{} system uses the Calculus of (co-)inductive
constructions as its underlying programming language and thus (by
Curry-Howard) as its logic. This system is immensively powerful and
can encode large parts of traditional matematics. However, there are
certain properties which are unprovable in the C(co-)iC.

An unprovable property should not let an aspiring \coq{} user down. We
can just add that property as an axiom to the system and take it for
granted. We must be very careful not to add a property to the system
which makes the system inconsistent though. It is known,
\cite{coq-add} section 5.2, that one can add the properties of \emph{Proof
  irrelevance} and \emph{Extensionality} simultaneously without
introducing any inconsistencies.

\paragraph{Proof Irrelevance}
\label{sec:proof-irrelevance}

The Irrelevance of proofs state that if we have two ways of proving
the same thing, it is irrelevant which one we look at. They are
decided as being equal. Formally:
\begin{axm}
  Suppose $P$ is any proposition. Let $p_1$ and $p_2$ be proofs of
  proposition $P$. In \coq{} this means $p_1$ and $p_2$ are terms of
  type $P$. Then $p_1 = p_2$.
\end{axm}

\paragraph{Extensionality}
\label{coqext:extensionality}

The extensionality axiom states that if two functions agree on all
values, then they are equal. Formally:
\begin{axm}
  Suppose we have $f, g \colon A \to B$. Then if $\forall x \in A,
  f(x) = g(x)$ we have $f = g$.
\end{axm}
Our development uses extensionality the most. It is used to determine
equality of (matematical) functions by discrimination of each element
in the function. Without this extension, we would have to resort to
another encoding of the JANUS store, as we shall see later on.

%%% Local Variables: 
%%% mode: latex
%%% TeX-master: "master"
%%% End: 

\fixme{Read through J0, correcting.}
\chapter{\janusz{}}
\label{chap:janus0}

In this chapter, our aim is to introduce a simple subset of the JANUS
language and provide the foundations for the \coq{} framework. The
\janusz{} programming language is a subset of the full JANUS language
from \cite{yokoyama.gluck:reversible}. It not a Turing Complete language and
contains only simple linear constructs. However, it contains enough to
provide a vehicle for explaining the basics of a formalization. We aim
to provide a big step operational semantics for the language, suitable
for encoding into the logical framework \coq{}.

Describing a language requires us to define the basic objects that are
manipulated. There are two such objects in \janusz{}: integers and
stores.

\paragraph{Integers}

The integers are the mathematical integers, ie drawn from $\ZZ$. In
full JANUS the primary object are 32-bit unsigned integers but for
simplification purposes, we omit these.
\begin{defn}
  \label{defn-lift}
  For any set, $S$, we define its domain-theoretic \emph{lift} to be
  $S_{\perp} = S \cup \{\perp\} $ for a special value $\perp$ called
  ``bottom''. Values $s \in S$ are notated as $\lift{s}$.
\end{defn}
Information-theoretically, the bottom values represents ``no
information'' whereas the lift of a value represents that value. This
is akin to the well-known ML datatype ``option''.

\paragraph{Stores}
The stores, notated as $\sigma, \sigma', \dotsc$ are functions from
natural numbers to lifted integers: $\sigma \colon \NN \to
\ZZ_{\perp}$. The elements of the domain are called
\emph{locations}. We usually give them names $x, y, z, \dotsc$ and
assume each variable is associated with a specific number for the
duration of the program. The co-domain defines the contents of the
current location in the store. These are called \emph{values}. If no
value is associated with the location $x$ then $\sigma(x) = \perp$ and
otherwise $\sigma(x) = \lift{i}$ for some $i \in \ZZ$. The process of
applying a location $x$ to a store $\sigma$ is called \emph{lookup}.

Stores are altered under the course of running a program. In our
formalization this amounts to changing the function
$\sigma$. The notation $\sigma[x \mapsto k]$ means ``update the contents
of location $x$ with the value $k$''. The mathematical formal meaning
is:
\begin{equation*}
  \sigma[x \mapsto k](z) = \begin{cases}
    \lift{k} & \quad \text{if} \; x = z\\
    \sigma(z)  & \quad \text{otherwise}
  \end{cases}
\end{equation*}

The empty store, $\epsilon$, maps everything to $\perp$, ie. for all
locations $l \in \ZZ$ we have $\epsilon(l) = \perp$. We will use the
empty store as the initial store.

It will be beneficial to provide \emph{hiding} of values in the store. A
hiding of a variable $x$ is a new store, in which $x$ does not
exist. Because we have the value $\perp$ in our store-values, there is
a straightforward definition. We use the notation $(\sigma \setminus x)$
for the hiding of the variable $x$ in the store $\sigma$. The definition
is:
\begin{equation*}
  (\sigma \setminus x)(z) = \begin{cases}
    \perp & \quad \text{if} \; z = x\\
    \sigma(z) & \quad \text{otherwise}
  \end{cases}
\end{equation*}

\subsection{Coq encoding}

One encodes the semantics of object languages in the language of
\gallina{}, the specification language of \coq{}. \gallina{} is
inherently functional and is furthermore \emph{total} for its
inductive part\footnote{We are not using the Co-Inductive features of
  \coq{}}. Functional programmers should feel at ease when working
with the language as it works like a normal programming language. But
since the language is based on the Calculus of (Co-)Inductive
constructions, the type system is far more advanced than one a
functional programmer is accustomed to. For \coq{}, version 8.2, you
will find most of the advanced type concepts directly in the type
system. If not, a concept is usually derivable. \coq{} supports
dependent types, the polymorphic lambda calculus, higher order types,
Haskell-style type classes, ML style module system with functors,
GADTs etc \cite{pierce:advanced, pierce:types, hall.hammond.ea:type,
  milner.tofte.ea:definition}.

\coq{} already provides a formalization of mathematical integers,
based on an infinite binary representation. This means we can focus
entirely on providing an implementation of stores.

In the stores given above, we operate with a domain of $\NN$ and a
co-domain of $\ZZ$ used as the locations and values respectively. For
encoding this in \coq{}, we define a \texttt{Module Type}. For ML
people, this is a type signature. The implementation is:
\begin{verbatim}
Module Type STORE.

  Parameter location : Set. (* Domain of the store mapping *)
  Parameter value : Set.    (* Codomain of the store mappring *)

  (* locations have equality *)
  Parameter eq : location -> location -> Prop.
  Parameter location_eq_dec :
      forall (n m : location), {n = m} + {n <> m}.

End STORE.
\end{verbatim}
In this definition we define two parameters, locations and
values. Both belong to the ``Set'' world, which is the world of
objects in \coq{}. The \texttt{eq} parameter specifies an equality
predicate on locations. It returns its result as a ``Prop'' which is
the world of propositions in the universe. The distinction between
``Set'' and ``Prop'' is made such that we keep the language and its
properties apart in the formalization, although the two universes are
the same underneath.

Finally, we require the existence an equality decider for
locations. It states that either $n = m$ or $n \neq m$ for locations
$n, m$.

Memories are defined as a \emph{Module Functor} expecting a
\texttt{STORE} fulfilling module as input and producing a
\texttt{Memory} as output. By using a functor we hoist the specifics
of the domain and co-domain of the memory, so we can change the
definition later on.

The memory is a function from locations to values $S_{\perp}$, encoded
as an option type as known from ML. This encodes a lift as given in
Definition \ref{defn-lift}. The value of $\perp$ is encoded as
\texttt{None} and the lift $\lift{s}$ as \texttt{Some s}. The
definition of the empty store is then straightforward, where locations
are named ``var'' for convenience:
\begin{verbatim}
  Definition empty (_ : var) : option value := None.
\end{verbatim}
Lookup on a memory is function application of the location to the
memory function. Writing a new value to a given location is happening
according the update function given above:
\begin{verbatim}
  Definition write (m : memory) x v x' :=
    if location_eq_dec x x'
      then Some v
      else m x'.
\end{verbatim}
Hiding is also carried out according to the mathematical definition.

\subsection{Properties of stores}

The stores we have defined has a set of properties associated with
them. These properties are important when we want to prove theorems
about \janusz{} as they form the \emph{Knowledge basis} for the
stores. The knowledge basis is the set of properties we build on when
we formalize properties of \janusz{}, but are not directly
related. When humans carry out proofs there is an implicit temptation
to assume the existence of most of this work by intuition. For a
machine however, we will have to provide it with the theorems as well
as the proofs. The proofs presented here is a selection from the
development.

\begin{lem}
  \label{lem:write-eq}
  For a store $\sigma$ we have $\sigma[x \mapsto v](v) = \lift{v}$
\end{lem}
\begin{proof}
  In \coq{} (case analysis on $x$ after unfolding the definition of updates).
\end{proof}
Theorem \ref{lem:write-eq} has a similar proposition:
\begin{lem}
  Assume a store $\sigma$ and let $x, y$ be locations with $x \neq
  y$. Then we have, for any value $v$:
  \begin{equation*}
    \sigma[x \mapsto v](y) = \sigma(y)
  \end{equation*}
\end{lem}
\begin{proof}
  By \coq{}.
\end{proof}

Naturally, the above statement can be extended to equalities, which
will be used later on:
\begin{lem}
\label{lem:write-eq2}
  We have:
  \begin{multline*}
    \forall \sigma, \sigma' \in \Sigma, x \in Loc, v_1, v_2 \in Value
    \colon \\
    \sigma[x \mapsto v_1](x) = \sigma[x \mapsto v_2](x) \implies v_1 = v_2
  \end{multline*}
\end{lem}
\begin{proof}
  By \coq{}
\end{proof}

The following property states that ``if you have enough pieces placed
correctly you have the whole puzzle''; that is, if you enough about a
store, you can decide its equality:
\begin{lem}
\label{lem:hide-eq}
  Assume two arbitrary stores $\sigma, \sigma' \in \Sigma$. Assume a
  variable $x$ and a value $v$ from any value domain. Now suppose
  \begin{itemize}
  \item $\sigma(x) = v$
  \item $\sigma'(x) = v$
  \item $(\sigma \setminus x) = (\sigma' \setminus x)$
  \end{itemize}
  Then $\sigma = sigma'$
\end{lem}
\begin{proof}
  By \coq{}.
\end{proof}

Another lemma on stores establishes a property when we are hiding
information:
\begin{lem}
\label{lem:hide_ne}
  Assume a store $\sigma \in \Sigma$ and two variables $x$ and
  $x'$. If $x \neq x'$ then we have $(\sigma \setminus x)(x') = \sigma(x')$.
\end{lem}
\begin{proof}
  In \coq{}, by the definition of hiding and case analysis.
\end{proof}

The following lemma will come in handy later. It is based on hiding
stores:
\begin{lem}
\label{lem:write_hide}
  Assume two stores $\sigma, \sigma' \in \Sigma$, a variable $x$ and
  two values $v_1, v_2 \in V$, for any value domain $V$. Then
  $\sigma[x \mapsto v_1] = \sigma'[x \mapsto v_2]$ implies $(\sigma
  \setminus x) = (\sigma' \setminus x)$
\end{lem}
\begin{proof}
  In \coq{}. The proof relies on backwards reasoning. It uses
  extensionality (see \ref{coqext:extensionality}) and then it uses
  case analysis. In the cases, it applies lemma \eqref{lem:hide_ne}.
\end{proof}

Another very important concept is that a write to the hidden variable
does not matter. The following lemma makes this specific
\begin{lem}
\label{lem:hide-write}
  Let there be given an arbitrary store $\sigma$, variable $x$ and value
  $v$. Then the following holds:
  \begin{equation*}
    (\sigma[x \mapsto v] \setminus x) = (\sigma \setminus x)
  \end{equation*}
\end{lem}
\begin{proof}
  The proof utilizes extensionality to discern each possible element
  in the maps. Then it discriminates on whether we are working with $x$
  or not. For each case, a simple computation discharges the sub-problem.
\end{proof}

\subsection{History}

In the course of construction the right store, we tried several
experiments which all proved to be less useful than the development
given here. At the beginning, we envisioned the store to be a map from
$Loc \to S$. The empty store were then defined as mapping everything
to the value $0$. The problem with this solution is that we cannot
differentiate between the value $0$ and no value at all. This
eliminates our knowledge of a location being invalid.

We ultimately changed to the store representation given, when it
became apparent we needed the ability to \emph{hide} certain store
locations. An ``option''-type was then the only right implementation.

\section{Expressions in \janusz{}}

The \janusz{} language has a very simplified expression language
compared to full JANUS. In this language there are 5 expression
constructs: integer constants, store referencing, addition,
subtraction and multiplication.

The syntax of expressions $e$ is the following in BNF notation:
\newcommand{\bor}{\; | \;}
\begin{equation*}
  e ::= n \bor \mathtt{x} \bor e + e \bor e - e \bor e * e
\end{equation*}
The judgement forms are $\sigma |- e => z$ stating that under the
assumption of a store $\sigma$ the expression $e$ evaluates to the
integer $z$. The inference rules for this system is straightforward:
\begin{gather*}
  \inference[Const]{}{\sigma |- n => n} \quad \inference[Var]{\sigma(\mathtt{x}) =
    \lift{k}}{\sigma |- \mathtt{x} => k} \\
  \inference[Add]{\sigma |- e_1 => n_1 \quad \sigma |- e_2 => n_2 \quad
    n_1 + n_2 = n}{\sigma |- e_1 + e_2 => n} \\
  \inference[Sub]{\sigma |- e_1 => n_1 \quad \sigma |- e_2 => n_2 \quad
    n_1 - n_2 = n}{\sigma |- e_1 - e_2 => n} \\
  \inference[Mul]{\sigma |- e_1 => n_1 \quad \sigma |- e_2 => n_2 \quad
    n_1 * n_2 = n}{\sigma |- e_1 * e_2 => n}
\end{gather*}

There is another, denotational, semantics for expressions
however. This semantics are equal to the operational semantics given
above. One aspect of Coq is that it is more natural to express
denotational semantics than operational semantics.

\label{exp:denot-semantics}
For a denotational semantics, we define a computation
$\mathcal{E}|[-|] \colon E \to \Sigma \to \ZZ_{\perp}$ from an
expression and a Store $\Sigma$ to a lifted value. The definition is a
case analysis on the structure of the expression given:
\begin{align*}
  \mathcal{E}|[n|](\sigma) & = \lift{n}\\
  \mathcal{E}|[x|](\sigma) & = \sigma (x)\\
  \mathcal{E}|[e_1 + e_2|](\sigma) & = \mathcal{E}|[e_1|](\sigma) \;
  +_{\ZZ} \;
  \mathcal{E}|[e_2|](\sigma)\\
  \mathcal{E}|[e_1 - e_2|](\sigma) & = \mathcal{E}|[e_1|](\sigma) \;
  -_{\ZZ} \;
  \mathcal{E}|[e_2|](\sigma)\\
  \mathcal{E}|[e_1 * e_2|](\sigma) & = \mathcal{E}|[e_1|](\sigma) \;
  *_{\ZZ} \;
  \mathcal{E}|[e_2|](\sigma)
\end{align*}
The operation $+$ is the operation from our expression language
whereas the operation $+_{\ZZ}$ is the mathematical integer addition
here made explicit by its annotation. Likewise is the case for $-$ and
$*$.

The advantage of the latter, denotational, definition is that is
allows for simpler proofs in Coq. Basically a standard Case analysis
will do over the structure. Operational semantics uses a Prolog-style
where a relation between the premises and conclusion is defined. The
advantage of this Prolog style is of course its generality. The
denotational style above is a function definition based on case
analysis which is a special case of a relation.

If $R \subseteq X \times Y$ is a relation on $X$ and $Y$, then the
relation would be a function in the following case: if $(a, b) \in R$
and also $(a, c) \in R$ then we must have $b = c$. In other words, a
function is deterministic in its output based on its output.

\subsection{Encoding expressions in \coq{}}

The syntax of \janusz{}-expressions is encodable as a normal datatype
known from e.g. ML or Haskell. These are called inductive types in
\coq{} and happens to be much more powerful than their ML
counterparts. Here, we will only use the ML-subset though.
\begin{verbatim}
    Inductive Exp : Set :=
    | Exp_Const : Z -> Exp
    | Exp_Var : Var -> Exp
    | Exp_Add : Exp -> Exp -> Exp
    | Exp_Sub : Exp -> Exp -> Exp
    | Exp_Mul : Exp -> Exp -> Exp.
\end{verbatim}

The denotational expression evaluation of \janusz{} can be encoded as a
\texttt{Fixpoint}. We have to show that the fixpoint is total, but
this is rather simple as we always decrease the size of the expression
when we evaluate it. Here we give the start of the definition and
explain it:
\begin{verbatim}
    Fixpoint denote_Exp (m : ZMem.memory) (e : Exp) {struct e}
        : option Z :=
      match e with
        | Exp_Const z => Some z
        | Exp_Var x => m x
        | Exp_Add e1 e2 =>
            match (denote_Exp m e1, denote_Exp m e2) with
              | (Some n1, Some n2) => Some (n1 + n2)
              | _ => None
            end
        ...
\end{verbatim}
This describes expression annotations as a fixpoint (recursive
function) taking a memory, and an expression as inputs. It produces a
value of type $\ZZ_{\perp}$ as output. The
\texttt{struct}-designation defines the input that gets gradually
smaller on recursion, which \coq{} will utilize for obtaining a
termination proof.

The function itself is a simple case match on the different
constructors of the inductive type. Certain constructors have been
omitted in the above fragment.

\subsection{Properties of \janusz{} expressions}

We will only need the forward determinism for expressions and hence
there is no need for the generality of the relation. We therefore
employ a denotational definition as it greatly simplifies the needed
proof of forward determinism:

\begin{thm}
  \janusz{} expressions are forward deterministic: suppose we have a
  store $\sigma$ and an expression $e$. Now let
  $\mathcal{E}|[e|](\sigma) = v_1$ for a value $v_1 \in \ZZ_{\perp}$
  and also let $\mathcal{E}|[e|](\sigma) = v_2$ for a value $v_2 \in
  \ZZ_{\perp}$. Then we have $v_1 = v_2$.
\end{thm}
\begin{proof}
By the use of \coq{} and the \texttt{grind}-tactic from
\cite{chlipala:certified}:
\begin{verbatim}
    Theorem exp_determ : forall m e v1 v2,
      denote_Exp m e = v1 -> denote_Exp m e = v2 -> v1 = v2.
    Proof.
      grind.
    Qed.
\end{verbatim}
\end{proof}
The proof of determinism shows the benefit of using a denotational
semantics for expressions. Since we are essentially defining our
function to be total in the language of the CoIC used by Coq, we gain
forward determinism for free by borrowing it from the underlying
language.

\section{Statements in \janusz{}}

The statement syntax of \janusz{} defines a small, purely linear
subset of the full JANUS language. There are 4 constructs given via
the following notation in BNF:
\reservestyle{\command}{\mathbf}
\command{if[\;if\;],then[\;then\;],else[\;else\;],fi[\;fi\;],skip[\;skip\;],
 +=[\;+\!\!=\;],-=[\;-\!\!\!=\;],call[\;call\;],uncall[\;uncall\;]}
\begin{gather*}
  s ::= \quad \<skip> \bor x \<+=> e \bor x \<-=> \bor s; s
  \bor \<if> e \<then> s \<else> s \<fi> e
\end{gather*}
\newcommand{\angel}[1]{\langle #1 \rangle}

We will interleave the description of each of these syntactical
elements with the inference rules for them. The judgement form for
execution of statements is $\angel{\sigma, s} -> \sigma'$ which
designates that under the assumption of a store $\sigma$, execution of
statement $s$ will yield the altered store $\sigma'$.

The very first operation is a simple one. The \textbf{skip} operation
is a no-operation command that simply skips. It has the following
rule:
\begin{equation*}
  \inference[Skip]{}{\angel{\sigma, s} -> \sigma}
\end{equation*}

The second operation is the increment operation of an element in the
store. This operation, written as $x +\!\!= e$ will evaluate the
expression $e$ to a number $k$ and then add this amount to the
location in the store to which $x$ points:
\begin{equation*}
  \inference[Inc0]{\sigma |- e => k \quad \sigma(x) = \lift{k'} \quad k +
    k' = n}{\angel{\sigma, x +\!\!= e} -> \sigma[x \mapsto n]}
\end{equation*}
In JANUS it is a requirement that the variable $x$ must not occur in
the expression $e$. The same requirement is present in \janusz{} and
has to do with the invertibility of such statements. There is,
however, an alternative semantics not present in the current
literature which directly encodes the requirement in the inference
rule.

Recall we defined an ability to ``hide'' certain variables in our
store. We can utilize this by hiding $x$ in the expression evaluation:
\begin{equation*}
  \inference[Inc]{(\sigma \setminus x) |- e => k \quad \sigma(x) =
    \lift{k'} \quad k + k' = n}{\angel{\sigma, x +\!\!= e} -> \sigma[x \mapsto n]}
\end{equation*}
Now, because expression evaluation of the ``Var'' case requires a
lifted value it is now impossible to construct an inference tree where
the expression $e$ refers to the value $x$. We have effectively
encoded the informal requirement into a formal one. This method,
correctness by construction, simplifies proof formalization: had we
chosen a predicate judgement for an non-occurring variable $x$, then
we would have to provide a proof with this added structure.

For decrementing, the definition is similar:
\begin{equation*}
  \inference[Dec]{(\sigma \setminus x) |- e => k \quad \sigma(x) =
    \lift{k'} \quad k - k' = n}{\angel{\sigma, x -\!\!= e} -> \sigma[x \mapsto n]}
\end{equation*}

The next rule is for sequencing operations. In the statement $s_1;
s_2$ one first executes $s_1$ and then feeds the resulting store into
the execution of $s_2$:
\begin{equation*}
  \inference[Seq]{\angel{\sigma, s_1} -> \sigma'' \quad
    \angel{\sigma'', s_2} -> \sigma'}
  {\angel{\sigma, s_1; s_2} -> \sigma'}
\end{equation*}

Finally, there is the rule for the branching instruction. In \janusz{}
the value $0$ is ``false'' and any value different from $0$ is the
``true'' value. This yields two rules, one for each case. The first
rule is for the ``false'' case:
\begin{equation*}
  \inference[If-false]{\sigma |- e_1 => 0 \quad \angel{\sigma, s_2} -> \sigma'
    \quad \sigma' |- e_2 => 0}{\angel{\sigma, \<if> e_1 \<then> s_1 \<else> s_2 \<fi> e_2} -> \sigma'}
\end{equation*}
This rule states that if the $e_1$ \emph{test} evaluates to a false
value, then the ``else''-branch is executed. Finally the
\emph{assertion} $e_2$ must be false as well.

The true case is similar, the difference being extra (mathematical)
requirements on what the expressions evaluates to:
\begin{equation*}
  \inference[If-true]{\sigma |- e_1 => k \quad k \neq 0 \quad
    \angel{\sigma, s_1} -> \sigma'
    \quad \sigma' |- e_2 => k' \quad k' \neq 0}{\angel{\sigma, \<if> e_1 \<then> s_1 \<else> s_2 \<fi> e_2} -> \sigma'}
\end{equation*}

The need for assertions has to do with reversibility, as we shall see
later on.
\subsection{Encoding statements in Coq}

Encoding of the statement syntax is rather straightforward. We use an
inductive type to capture the valid syntax rules. We omit it here for
brevity as it is pretty simple and uninteresting.

Encoding the inference rule system for \janusz{} is more
interesting. To do this, we also use an inductive type, but make it a
member of the ``Prop'' world of propositions. Furthermore, we use a
\emph{dependent} type to build a family of types for inference
rules \cite{pierce:advanced}.

We make \texttt{Stm\_eval} inductive with the type signature $mem \to
Stm \to mem \to \mathbf{Prop}$ which makes it dependent on the input
memory, the statement executing and the resulting memory. This is done
in \coq{} by:
\begin{verbatim}
Inductive Stm_eval : mem -> Stm -> mem -> Prop :=
...
\end{verbatim}

Next, we encode each inference rule as a constructor of this inductive
type. For instance, we encode the skip rule as:
\begin{verbatim}
| se_skip: forall m, Stm_eval m S_Skip m
\end{verbatim}
This states that for all memories $m$, we have that skip yields $m$ as
the resulting memory. Think about it as a Prolog-style relation on
statement evaluation.

Increment uses some premises:
\begin{verbatim}
| se_assvar_incr: forall (m m': mem) (v: Var)
   (z z' r: Z) (e: Exp),
  denote_Exp (ZMem.hide m v) e = Some z ->
  m v = Some z' ->
  r = (z + z') ->
  m' = ZMem.write m v r ->
    Stm_eval m (S_Incr v e) m'
\end{verbatim}
Note how each premise is encoded as an assumption of the
constructor. If all premises are fulfilled, then we have a relation on
$m$, $\text{S\_Incr}\; v\; e$ and $m'$. The other rules are similar.

\subsection{Proving determinism of \janusz{}}

For \janusz{}, we will prove a couple of theorems. The 2 main theorems
to be proven are those of forward and backward determinism. The
language must be deterministic in the forward direction, so there is
only one possible outcome of any computation. Also, it must be
deterministic in the backwards direction: if we have a computation
resulting in a resulting store, the original store must be equivalent.

It is clear this establishes a necessary condition for
reversibility. Had the language not been backwards deterministic, it
would be outright impossible to reconstruct the input store from the
output store.
\begin{lem}
\label{j0-fwd-det-prime}
  Let $\sigma, \sigma' \in \Sigma$ be stores. Let $s$ be any \janusz{}
  statement. Now, if $\angel{\sigma, s} -> \sigma'$ implies for all $\sigma''
  \in \Sigma$ we have $\angel{\sigma, s} -> \sigma''$, then $\sigma' =
  \sigma''$. Formally:
  \begin{equation*}
    \forall \sigma, \sigma' \in \Sigma, s \in Stm \colon
    \angel{\sigma, s} -> \sigma' \implies (\forall \sigma'' \in \Sigma
    \colon \angel{\sigma, s} -> \sigma'' \implies \sigma' = \sigma'')
  \end{equation*}
\end{lem}
\begin{proof}
  In \coq{}.
\end{proof}

The proof proceeds by induction over the 1st dependent hypothesis, ie
$\angel{\sigma, s} -> \sigma'$. The increment and decrement cases are
similar. They proceed by noting (for increment) that $(\sigma
\setminus x) |- e => k$ occurs in the premise of $\angel{\sigma, s} ->
\sigma''$ as well, so the $k$ must be identical. The same is the case
for $\sigma(x) = \lift{k'}$ and hence $k + k'$ is written in both
cases.

The semi-case just aspires to the induction hypothesis. Due to the way
the lemma is worded we have $\forall \sigma'' \colon \angel{\sigma, s}
-> \sigma''$ as the induction hypothesis and then this is trivially
proven.

The if-case is done by inversion of $\angel{\sigma, \<if> e1 \<then> s1
\<else> s2 \<fi> e2} -> \sigma''$. One of these are solvable by the
induction hypothesis. The other is solvable because it is an
impossible case (false-inversion occurring while proving the if-true
subgoal -- or true-inversion when proving the if-false subgoal).

\begin{thm}
\label{thm:j0-fwd-det}
  \janusz{} is forward deterministic, ie:
  \begin{equation*}
    \forall \sigma, \sigma', \sigma'' \in \Sigma, s \in Stm \colon
    \angel{\sigma, s} ->
    \sigma' \implies \angel{\sigma, s} -> \sigma''
    \implies \sigma' = \sigma''
  \end{equation*}
\end{thm}
\begin{proof}
  The $\forall \sigma''...$ is hoistable in lemma
  \eqref{j0-fwd-det-prime} due to logic rules. This is made formal in
  \coq{} by introducing everything but $\sigma' = \sigma''$ as as
  hypotheses and then generalizing over the form needed to apply lemma
  \eqref{j0-fwd-det-prime}.
\end{proof}

In the forward determinism-proof nothing special has been applied to
carry out the proof. Either one can calculate (in the integer ring
$\ZZ$) and conclude the result by identification or one can apply the
induction hypothesis to obtain the result after inversion. As we shall
see, the backwards deterministic proof is not as simple, though it
holds:

\begin{lem}
  Let $\sigma, \sigma' \in \Sigma$ be stores. Let $s \in Stm$ be any
  statement. Assume $\angel{\sigma', s} -> \sigma$ (note the position of
  $\sigma'$, compare with lemma \eqref{j0-fwd-det-prime}). Then
  $\forall \sigma'' \colon \angel{\sigma'', s} -> \sigma$ implies $\sigma' =
  \sigma''$. Formally:
  \begin{equation*}
    \forall \sigma, \sigma' \colon \angel{\sigma', s} -> \sigma \implies
    (\forall \sigma'' \colon \angel{\sigma'', s} -> \sigma \implies \sigma'
    = \sigma'')
  \end{equation*}
\end{lem}
\begin{proof}
  Via \coq{}
\end{proof}

\fixme{Supply an example of how to use \coq{} tactics here}
Completing this proof requires considerably more ingenuity than the
forward proof. Again, we run induction on the 1st dependent
hypothesis. For the increment case, we first run inversion, and
identify equal terms via the \texttt{subst} tactic. We must have
$(\sigma' \setminus x) = (\sigma'' \setminus x)$ because we have
$\sigma'[x \mapsto k + k'] = \sigma''[x \mapsto k_0 + k_0']$ (lemma
\eqref{lem:write_hide}). This means, that $k = k_0$. Next, we have $k
+ k' = k_0 + k_0'$. This is because of lemma \eqref{lem:write-eq2} and
analysis on $\sigma'[x \mapsto k + k'] = \sigma''[x \mapsto k_0 +
k_0']$. Computation in the ring $\ZZ$ now gives that $k' =
k_0'$. After substitution, we then use lemma \eqref{lem:hide-eq}
proving the case. The decrement case is equivalent.

The cases using the induction hypothesis (semi/if) are trivial,
utilizing the \texttt{congruence} tactic to solve the impossible cases
in the evaluation of the ``if'' cases.

\begin{thm}
  \janusz{} is backward deterministic, ie:
  \begin{equation*}
    \forall \sigma, \sigma', \sigma'' \in \Sigma, s \in Stm \colon
    \angel{\sigma', s} -> \sigma \implies \angel{\sigma'', s} -> \sigma \implies \sigma' = \sigma''
  \end{equation*}
\end{thm}
\begin{proof}
  Like the proof of theorem \eqref{thm:j0-fwd-det}, we can hoist the
  forall-quantifier and formalize it in \coq{}.
\end{proof}

\section{Inversion of \janusz{}}

In order to invert \janusz{}, we define a rewriting system
$\mathcal{I}(\cdot) \colon Stm \to Stm$ from statements to
statements. The rewriting system is straightforwardly defined: note
that each atomic operation has a in inverse in the language and that
each compound statement can be inductively inverted:
\begin{align*}
  \mathcal{I}(\<skip>)& = \<skip>\\
  \mathcal{I}(x \<+=> e)& = x \<-=> e\\
  \mathcal{I}(x \<-=> e)& = x \<+=> e\\
  \mathcal{I}(s1; s2)& = \mathcal{I}(s2); \mathcal{I}(s1)\\
  \mathcal{I}(\<if> e_1 \<then> s_1 \<else> s_2 \<fi> e_2)& = \<if> e_2 \<then> \mathcal{I}(s_1) \<else> \mathcal{I}(s_2) \<fi> e_1
\end{align*}

The first sanity-check on the rewriting system is that of
self-isomorphism; applying the rewriting system twice yields the
original:
\begin{thm}
  For any statement $s$, we readily have:
  \begin{equation*}
    \mathcal{I}(\mathcal{I}(s)) = s
  \end{equation*}
\end{thm}
\begin{proof}
  By induction over $s$ in \coq{}.
\end{proof}

In the paper \cite{yokoyama.gluck:reversible} an equivalence relation, $e_1 \sim e_2$,
is created between expressions $e_1$ and $e_2$ by
\begin{equation*}
  e_1 \sim e_2 \equiv \forall v \in Value, \sigma \in \Sigma \colon
  \sigma |- e_1 => v <=> \sigma |- e2 => v
\end{equation*}
Similarly, a relation $s_1 \sim s_2$ is created between statements
$s_1$ and $s_2$ by
\begin{equation*}
  s_1 \sim s_2 \equiv \forall \sigma, \sigma' \in \Sigma \colon
  \angel{\sigma, s_1} -> \sigma' <=> \angel{\sigma, s_2} -> \sigma'
\end{equation*}

We have defined these in \coq{} for \janusz{} and shown them to be
equivalence relations indeed. It can then be proven that the
sequencing in \janusz{} is associative: $s_1; (s_2; s_3) \sim (s_1;
s_2); s_3$. This proof was carried out in \coq{}. In the paper
associative property among others is used to prove the following theorem:
\begin{thm}
  Let $s$ be an arbitrary \janusz{} statement and let $\sigma$ and
  $\sigma'$ be arbitrary stores. The following is then true:
  \begin{equation*}
    \angel{\sigma, s} -> \sigma' \iff \angel{\sigma', \mathcal{I}(s)} -> \sigma
  \end{equation*}
\end{thm}

At least for \janusz{} however, you don't need the equivalence
relation at all. Perhaps it is needed when running induction on the
statement $s$ as is done in the paper, but another path is to run a
generalized induction hypothesis on $\forall \sigma, \sigma' \in
\Sigma, s \in Stm, \angel{\sigma, s} -> \sigma' \implies
\angel{\sigma', \mathcal{I}(s)} -> \sigma$ (and the converse in the other
direction). This allows you to apply the induction hypothesis in the
sequence case \texttt{semi} in the obvious way and stitch the result
together. The hardest part of proving this statement in \coq{} is to
get the math right in the inversion of $\<+=>$ and $\<-=>$. You need
several properties of stores, including lemmas \eqref{lem:hide-write},
\eqref{lem:hide-eq} and \eqref{lem:write-eq} to mangle the store into
place.
\begin{proof}
  In \coq{}.
\end{proof}

%%% Local Variables: 
%%% mode: latex
%%% TeX-master: "master"
%%% End: 

\fixme{Read through J1, correcting.}
\chapter{\januso{}}
\label{chap:janus1}

In this chapter we will extend the \janusz{} language with new
concepts, forming the language \januso{}. The \januso{} language adds
several extra inference rules to the \janusz{} language in order to
make it more in par with the full complete language
specification. Concretely, we add:
\begin{itemize}
\item 32-bit unsigned integers with wrap-around arithmetic $(\mod
  2^{32})$.
\item Procedure calls
\end{itemize}
\januso{} only lacks arrays and loops from the full Janus language.

We first introduce 32-bit integers in \coq{} and then use the
formalization of these in the development of \januso{}.

\section{32-bit Integers}
\label{sec:32-bit-integers}

The integers chosen in JANUS are as usual machine words on a 32-bit
machine. Most traditional computers have an upper limit to what
numbers it can represent. A 32-bit machine has $32$ bits at its
disposal so it can represent $2^{32}$ different values. A 64-bit
machine has $2^{64}$ different values and so on. For the case where
there is no sign, it is obvious to map the representation into $0,
\dotsc, 2^{32}-1$ and do calculation modulo $2^{32}$. Coincidentally,
this is how most modern computers work anyway for unsigned arithmetic.

JANUS chooses this representation as well and models a 32-bit machine
in the process. For \januso{}, we would like to model the same thing.

From the perspective of mathematics, 32-bit values means we are
working in the ring $\mathbb{Z}/2^{32}\mathbb{Z}$
\cite{jensen:klassisk}.  We could have chosen other
rings as the base for our numbers in JANUS.

Let us recap exactly what is meant by a ring
\begin{defn}
  A ring is a set $R$ equipped with two binary compositions $+\colon R
  \times R \to R$ and $\cdot \colon R \times R \to R$. The $+$
  composition forms an abelian group and the $\cdot$ composition forms
  a monoid. Furthermore, the ``multiplication'' distributes over the
  addition by the following rules:
  \begin{align*}
    a \cdot (b + c) & = (a \cdot b) + (a \cdot c)\\
    (a + b) \cdot c & = (a \cdot c) + (b \cdot c)
  \end{align*}
\end{defn}

One particular ring is well-known and a good candidate for a number
base; the usual ring of integers $\mathbb{Z}$. Implicitly,
this was the choice for \janusz{}. From a formal viewpoint, this ring
is bliss to work with. On the other hand, it means that the formal
machine have arbitrarily sized integers, which may be a bit
unrealistic. Hence, we chose numbers modulo $2^{32}$ as we can easily
formalize 64-bit numbers as well if needed by changing a constant. We
will call these integers $W^{32}$.

\begin{rem}
  Note that the ``multiplication'' need only form a monoid and not a
  group in general. We are thus not sure of a multiplicative inverse,
  which is somewhat bad. Remedying this could be to use finite
  Galois-fields, of order $p^n$ where $p$ is a prime and $n$ is an
  integer. For the case $p = p^1$ it is simple since it is defined by
  the groups $\ZZ / p\ZZ$ (for addition) and $(\ZZ / p\ZZ)^{*}$ (for
  multiplication). For the case $p^n$ for $n > 1$ you need to use
  splitting fields and more heavyweight algebra to define them. This
  is turn makes them rather bad candidates as a number base. Perhaps
  except for the base $2^n$ which can be defined as polynomials of
  bits. In any case, Galois-fields are not easily
  machine-implementible in hardware, so we choose not the pursue this
  idea further.
\end{rem}

\subsection{\coq{} Implementation}

The \coq{} implementation of 32-bit numbers owes everything to Xavier
Leroy et.al.\cite{leroy:compcert}. The work is taken from the
\textsc{CompCert} project, which introduces them in the course of
formalizing a complete C to PowerPC compiler. The reason for taking
the work rather than building it from scratch is due to the sheer
complexity of getting it right. The formalization uses standard
algebraic properties to systematically build the necessary knowledge
and properties in \coq{}.

We did some simplification to the full specification as we only needed
parts of it. We also added a couple of new properties to the
development which is needed for proving \januso{}.

We can't resist the temptation to describe the implementation however. A
32-bit number is a pair: An integer $i$ and a proof that $0 \leq i <
2^{32}$. The main theorem is:
\begin{lem}
  Let $x$ be in integer. For any such $x$ we have
  \begin{equation*}
    0 \leq (x \;mod\; 2^{32}) < 2^{32}
  \end{equation*}
\end{lem}
\begin{proof}
  From the \coq{} Development:
\begin{verbatim}
Lemma mod_in_range:
  forall x, 0 <= Zmod x modulus < modulus.
  intros.
  exact (Z_mod_lt x modulus (two_power_nat_pos word_size)).
Qed.
\end{verbatim}
  This states it is exactly given by the common result
  \texttt{Z\_mod\_lt} from the \coq{} standard library.
\end{proof}
The theorem allows us to take any integer and represent it as a 32-bit
number as long as we take the integer $(\mod 2^{32})$. Thus we have a
straightforward embed/project pair of functions between normal
integers and 32 bit integers.

For example, here is the definition of addition:
\begin{verbatim}
Definition add (x y: w32) : w32 :=
  repr (unsigned x + unsigned y).
\end{verbatim}
The function \texttt{unsigned} projects $x$ and $y$ to integers. Then
we integer-add the numbers and re-embed them with the \texttt{repr}
function. Note that the embedding has the correct behaviour: it wraps
around.

Bit-wise operations in the representation is implemented by creating
indicator functions from numbers up to a certain size in bits. Ie. a
number is converted into a function $f \colon \mathbb{Z} \to \{0,
1\}$, where $f(n) = 1$ indicates $n$-th bit is set and $f(n) = 0$ that
it is not. Straightforward combinations, projections, and embeddings
of these functions builds up the bit-wise functions.

The properties of the numbers are simply a direct (rather impressive)
work of standard algebra. The theorems can be found in any
introductory math-book on algebra, for instance
\cite{thorup:algebra}. First, we have
\begin{defn}
  Two integers $x, y$ are in the same congruence class modulo $m$ iff
  there exists an integer $k$ such that
  \begin{equation*}
    x = k \cdot m + y
  \end{equation*}
\end{defn}
With this definition Xavier and co. then proceeds to build up a
complete library of known algebraic properties for congruence classes
like the one above. As an example we present the addition of
congruence classes:
\begin{verbatim}
  Lemma eqmod_add:
    forall a b c d, eqmod a b -> eqmod c d
      -> eqmod (a + c) (b + d).
  Proof.
    intros a b c d [k1 EQ1] [k2 EQ2]; red.
    subst a; subst c. exists (k1 + k2). ring.
  Qed.
\end{verbatim}
If $a, b$ is a congruence class pair and $c, d$ also is, then the pair
$a+c, b+d$ is an equivalence pair (for a fixed modulo). The proof proceeds by the
well-known method that there must exist a $k_1$ by the first pair such
that $a = k_1 \cdot m + b$. Likewise $c = k_2 \cdot m + d$. Adding
these 2 equations and rearranging yields $a + c = (k_1 + k_2) \cdot m
 + (b + d)$ which is the form one wants. Most of the proofs reflect
usual mathematical methods like this.

\paragraph{Additions to the CompCert integers}

We added some additional properties to the \textsc{CompCert}
implementation of 32-bit integers. All the proofs of these are written
in \coq{} by the use of already given properties by Leroy
et.al. We extend their definitions by:
\command{xor[\;xor\;]}
\begin{lem}
  For any $x \in W^{32}$ we have
  \begin{equation*}
    x \<xor> x = 0
  \end{equation*}
\end{lem}
\begin{lem}
  For 32-bit integers $x,y,z$, additions can be discharged on the
  left:
  \begin{equation*}
    x + y = x + z \implies y = z
  \end{equation*}
\end{lem}
\begin{lem}
  For 32-bit integers $x, y, z$, subtraction can be discharged on the
  right:
  \begin{equation*}
    x - z = y - z \implies x = y
  \end{equation*}
\end{lem}
\begin{lem}
  For 32-bit integers $x, y, z$, xor can be discharged on left:
  \begin{equation*}
    x \<xor> z = y \<xor> z \implies x = y
  \end{equation*}
\end{lem}

\section{Augmenting expressions}

To the expressions of \janusz{}, we change the base type from $\ZZ$ to
$W^{32}$. Additionally, we make the following additions:
\begin{align*}
  e ::= & \dotsc \bor e / e \bor e \mod e \bor e = e \bor e \neq e \\
        \bor & e \land e \bor e \lor e \bor e < e
\end{align*}

The additions to the expression language are as in
\cite{yokoyama.axelsen.ea:principles}; we add arithmetic operators of
division and modulo to 32-bit numbers. We add a couple of relational
operators, namely equality, non-equality, logical and, logical or and
a less-than operator.

With these additions, we almost have all constructs from the full
JANUS language. It is our belief that the remaining operators can be
added without any problems. Specifically, our 32-bit integer
representation provisions for bit-wise operators.

The semantics are given by a denotational semantics in the spirit of
the denotational semantics in \ref{exp:denot-semantics}. We omit them
here for brevity and refer the interested reader to the \coq{}
development.

\section{Augmenting statements}

To the statement language we add the rules of \texttt{call},
\texttt{uncall} and xor ($\hat{=}$).
\begin{equation*}
  s ::= \dotsc \bor \<call> p \bor \<uncall> p \bor x \; \hat{=} \; e
\end{equation*}
where $p$ designates a procedure. We will see how we encode this
procedure shortly.

The semantical judgement forms of statements are altered to a form
$\rho |= \angel{\sigma, s} -> \sigma'$, where $\rho$ is a map from
$\ZZ \to Stm$. That is, we represent a procedure by an integer and its
value in the set of procedure definitions is a statement. We note that
all existing rules in the operational semantics just has to pass
$\rho$ on via congruence. Hence, we do not bring these rules here. The
rule of calling a procedure is
\begin{equation*}
  \inference[Call]{\rho |= \angel{\sigma, \rho(p)} -> \sigma'}
  {\rho |= \angel{\sigma, \<call> p} -> \sigma'}
\end{equation*}
Informally, the rule states that a procedure call looks up the
procedure in the context and then executes the procedure. Uncalling a
procedure, running the procedure in inverse, is defined as
\begin{equation*}
  \inference[Uncall]{\rho |= \angel{\sigma', \rho(p)} -> \sigma}
  {\rho |= \angel{\sigma, \<uncall> p} -> \sigma'}
\end{equation*}
In this rule, we utilize the full power of operational-semantics given
as a Prolog-style rule. Rather than relying on the inversion operator
we will define shortly, we alter the order procedure will run. This is
the main reason we choose an operational semantics rather than a
denotational one as the latter would have a harder time representing
this specific rule.

For the xor-operation, we use a rule which reflects the increment and
decrement operator:
\begin{equation*}
  \inference[Xor]{(\sigma \setminus x) |- e => \lift{k} \quad \sigma(x) =
    \lift{k'} \quad k \oplus_{32} k' = n}{\angel{\sigma, x \;\hat{=}\; e} -> \sigma[x \mapsto n]}
\end{equation*}
where $\oplus_{32}$ is the bit-wise xor-operation.

\section{Encoding expressions and statements}

The encoding of expressions and statements follows the same style as
we used earlier. The expression evaluation is extended with the new
constructors and utilizes our encoding of 32-bit words. We take the
existing encoding and gradually adapt it towards \januso{}.

Statements are implemented by first introducing a mapping $\ZZ \to
Stm$ which we then use for the $\rho$ in the above judgement
forms. Then the Inductive definition of statements are altered to take
this $\rho$ as and additional dependent type.

With these encodings, we can capture all of \januso{} and we can turn
our focus towards the properties of the language.

\section{Determinism properties of \januso{}}

Converting the proofs from \janusz{} to \januso{} has one main
complication: we went from $\ZZ$ to $W^{32}$ as our number type. When
\coq{} works with integers, and we only used Presburger arithmetic we
can solve such terms by the use of the \texttt{omega} tactic in
\coq{}. For our 32-bit integers, no such thing is present however, so
we need to do the proofs by hand.

The detail at which one has to carry out such proofs are almost
excruciatingly precise. We must, as an example, explicitly use the
commutative rule when we want to rewrite $x +_{32} y$ into $y +_{32}
x$.

\fixme{This section needs to be rewritten}
However, the formalization in \coq{} is slightly different to work
around a problem with mutually defined theorems. This means we cannot
utilize the theorem in further development as it is given in the
development.
\begin{thm}
\label{thm:j1-fwd-det}
  \januso{} is forward deterministic, ie:
  \begin{equation*}
    \forall \sigma, \sigma', \sigma'' \in \Sigma, s \in Stm \colon
    \rho |= \angel{\sigma, s} -> \sigma' => \rho |= \angel{\sigma, s} -> \sigma'' => \sigma' = \sigma''
  \end{equation*}
\end{thm}
\begin{proof}
  In \coq{}
\end{proof}
and
\begin{thm}
\label{thm:j1-bwd-det}
  \januso{} is backward deterministic, ie:
  \begin{equation*}
    \forall \sigma, \sigma', \sigma'' \in \Sigma, s \in Stm \colon
    \rho |= \angel{\sigma', s} -> \sigma => \rho |= \angel{\sigma'', s} -> \sigma => \sigma' = \sigma''
  \end{equation*}
\end{thm}
\begin{proof}
  Formalized in \coq{}
\end{proof}

\section{Inversion and its properties}

Inverting \januso{} works exactly like inverting \janusz{}. We only
need to add a couple of new rules for $\<call>$ and $\<uncall>$ and
the $\hat{=}$ operation. The $\<call>$ and $\<uncall>$ statements are
each others inverses and the $\<xor>$ operation is self inverse:
\begin{align*}
    \mathcal{I}(x \;\hat{=}\; e)& = x \;\hat{=}\; e\\
    \mathcal{I}(\<call> p)& = \<uncall> p\\
    \mathcal{I}(\<uncall> p)& = \<call> p
\end{align*}

With these rules, we can again prove the inversion sound by the
following theorem:
\begin{thm}
  Let $\rho$ designate an arbitrary \januso{} program, $s$ an
  arbitrary \januso{} statement and let $\sigma$ and $\sigma'$ be
  arbitrary stores. The following is then true:
  \begin{equation*}
    \rho |= \angel{\sigma, s} -> \sigma' \iff \rho |= \angel{\sigma', \mathcal{I}(s)} -> \sigma
  \end{equation*}
\end{thm}
\begin{proof}
  By \coq{}. The proof can be completed by straightforward extension of the
  same proof for \janusz{}.
\end{proof}


%%% Local Variables:
%%% mode: latex
%%% TeX-master: "master"
%%% End:

\fixme{Read through J, correcting.}
\chapter{Full JANUS}

\fixme{To be written}

Describe the semantics of full JANUS. Describe how one can do arrays
in the full language and how loops are encoded. Start with loops and
then think about arrays. Describe the problem with the backwards
determinism property. Describe and analyze why it is so.

...

%%% Local Variables: 
%%% mode: latex
%%% TeX-master: "master"
%%% End: 

\fixme{Read through FW, correcting.}
\chapter{Further Work}

This project leaves several venues by which one can continue. First of
all, some might want to finish the work started. This requires that
one devises a new semantics for loops and proves it to be
deterministic in both directions.

The loops in question must be created in such a way that we can use
discrimination to discharge the ``wrong'' cases in an inversion. Any
simple attempt to do this by the author failed, so it requires some
thought getting right.

Second, one might want to add arrays to the formalization. This is
certainly possible but requires some work getting right. We have
already laid out a general path for doing this.

Third, the paper \cite{glueck+2008} introduces a stack primitive into
the language. It could be interesting to formalize this part in \coq{}
as well.

Finally, we could formalize other reversible languages or instruction
set architectures. The hope is, like in the case of JANUS, that we
would gain some newfound knowledge on how to go on about formalizing
reversible languages.

%%% Local Variables: 
%%% mode: latex
%%% TeX-master: "master"
%%% End: 

\fixme{Read through Conc., correcting}
\chapter{Conclusion}
\fixme{write the conclusion}
Describe what we did in the project.
Reiterate our academic contributions.
Reiterate we fulfilled our learning goals.

%%% Local Variables: 
%%% mode: latex
%%% TeX-master: "master"
%%% End: 

\bibliography{biblio}
\appendix
%\input{../BaseLib}
\input{../Memory.tex}
\input{../ZStore}
\input{../Janus0}
\input{../Word32}
\input{../W32Store}
\input{../Janus1}

%%% Local Variables: 
%%% mode: latex
%%% TeX-master: "master"
%%% End: 

\end{document}