\chapter{Further Work}

We now describe how one could extend our work in new directions. This
project leaves several venues by which one can continue. First of all,
some might want to finish the work started. This requires that one
devises a new semantics for loops and proves it to be deterministic in
both directions.

The loops in question must be created in such a way that we can use
discrimination to discharge the ``wrong'' cases in an inversion. Any
simple attempt to do this by the author failed, so it requires some
thought getting right.

Second, one might want to add arrays to the formalization. This is
certainly possible but requires some work getting right. We have
already laid out a general path for doing this.

Third, the paper \cite{glueck+2008} introduces a stack primitive into
the language. It could be interesting to formalize this part in \coq{}
as well.

Finally, we could formalize other reversible languages or instruction
set architectures. The hope is, like in the case of JANUS, that we
would gain some new-found knowledge on how to go on about formalizing
reversible languages. An interesting language would be one that
utilizes a stack primitive and recursion to provide reversibility as a
variant of a forth-like language. Such a language could possibly be
interesting to formalize and show to have reversibility properties.

%%% Local Variables: 
%%% mode: latex
%%% TeX-master: "master"
%%% End: 
