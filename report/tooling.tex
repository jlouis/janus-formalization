\chapter{JANUS Primitives}

\fixme{Coq integration}
\fixme{Examples}

Before we can discuss the language of JANUS, we will need to take care
of the primitives used in the lanugage. Almost any formal treatment of
a language will start with a certain set of base primitives. In JANUS,
we have (single) base type of integers and a store.

It might seem wrong to go into great detail with the base
primitives. But it turns out that we can leverage features of the base
primitives in our formalization. If the primitives are chosen
correctly, it is possible to formalize the language more
precisely. Hence it is beneficial to design the primitives in a way
that simplifies, or even makes possible, a correct formalization.

When formalizing JANUS, we must take great care in not hiding facts in
the report text. A mechanical treatment must be understandable by the
computer, so we can not accept parts in natural language. Every device
needed for the formalization must exist in a machine readable (formal)
way. The primary driver for the base primitives is this mechanization:
we delibarately chose definitions we can use to our advantage in the
formalization.

\paragraph{Numbers}

The integers chosen in JANUS are modelled as usual machine words. In
most traditional computers the machine have an upper limit to what
numbers it can represent. A 32-bit machine has $32$ bits at its
diposal so it can represent $2^{32}$ different values. A 64-bit
machine has $2^{64}$ different values and so on. For the case where
there is no sign, it is obvious to map the representation into $0,
\dotsc, 2^{32}-1$ and do calculation modulo $2^{32}$. Coincidentially,
this is how most modern computers work anyway.

From the perspective of mathematics, it means we are working in the
additive group $\mathbb{Z}/2^{32}\mathbb{Z}$. We could have chosen
other groups as the base for our numbers in JANUS. In fact, any cyclic
additive group will do. If we choose the number to mod by as a prime
number, mulitplication will have an inverse as well\footnote{In fact
  we would have a mathematical field, $\mathcal{F}_p$}, but the idea
is not idiomatic with computers: they tend to use powers of $2$ for
their word size.

Let us recap exactly what is meant by a commutative group:
\begin{defn}
  An additive group is a set $G$ with a binary operator $+\colon G
  \times G \to G$ such that it fulfills the following properties:
  \begin{itemize}
  \item For $x,y,z \in G$ we have $(x + y) + z = x + (y + z)$.
  \item There exist an element $0 \in G$ such that for any element $x
    \in G$ we have $x + 0 = 0 + x = x$.
  \item For any element $x \in G$ there exist an \emph{inverse}
    element, notated $-x$, such that $x + -x = -x + x = 0$
  \item For elements $x, y \in G$ we have $x + y = y + x$
  \end{itemize}
  The first rule is called associativity; the second is the existence
  of a neutral element and the third is the existence of an inverse
  element. The fourth rule defines that the binary operation is
  commutative.

  If the fourth rule is omitted, we talk about a group. This is the
  normal definition to which one adds commutativity to get a
  commutative group (also called an abelian group).

  If both the fourth and the third rules are omitted, we have a
  \emph{monoid}, but these will not be interesting to our work as we
  need the inversion property.
\end{defn}

One particular abelian group is well-known and a good candidate for a
number base for JANUS; the usual group of integers $\mathbb{Z}$. From
a formal viewpoint, this group is bliss to work with. On the other
hand, it means that the formal machine we build have arbitrarily sized
integers, which may be a bit unrealistic. Hence, we chose numbers
modulo $2^{32}$ as we can easily formalize 64-bit numbers as well if
needed by changing a constant. We will call these integers $W^{32}$.

\subparagraph{Coq Implementation}

The Coq implementation of 32-bit numbers owes everything to Xavier
Leroy et al. The work is taken from the \textsc{CompCert} project as
it was deemed from the beginning it would be too hard to build a
number implementation ourselves. Had we chosen the group $\mathbb{Z}$,
then we would have been able to use the representation in the Coq
standard library.

But we can't resist the temptation to describe the implementation. A
32-bit number is a pair: An integer $i$ and a proof that $0 \leq i <
2^{32}$. The main theorem is:
\begin{thm}
  Let $x$ be in integer. For any such $x$ we have
  \begin{equation*}
    0 \leq (x \;mod\; 2^{32}) < 2^{32}
  \end{equation*}
\end{thm}
\begin{proof}
  From our Coq Development:
\begin{verbatim}
Lemma mod_in_range:
  forall x, 0 <= Zmod x modulus < modulus.
  intros.
  exact (Z_mod_lt x modulus (two_power_nat_pos word_size)).
Qed.
\end{verbatim}
  Which states that it is exactly given by the common result
  \texttt{Z\_mod\_lt} from the Coq standard library.
\end{proof}
The theorem allows us to take any integer and represent it as a 32-bit
number as long as we take the integer $\mod 2^{32}$. Thus we have a
straightforward embed/project pair of functions between normal
integers and 32 bit integers.

For example, here is the definition of addition:
\begin{verbatim}
Definition add (x y: w32) : w32 :=
  repr (unsigned x + unsigned y).
\end{verbatim}
The function \texttt{unsigned} projects $x$ and $y$ to integers. Then
we integer-add the numbers and re-embed them with the \texttt{repr}
function.

Bitwise operations in the representation works by creating
indicator-functions from numbers up to a certain size in bits. Ie a
number is converted into a function $f \colon \mathbb{Z} \to \{0,
1\}$, where $f(n) = 1$ if the $n$-th bit is set and $f(n) = 0$
otherwise. Straightforward combinations and embeddings of these
functions builds up the bitwise functions.

The properties of the numbers are simply a direct (rather impressive)
work of standard algebra. The theorems can be found in any math-book
on algebra. First, we have
\begin{defn}
  Two integers $x, y$ are in the same congruence class modulo $m$ iff
  there exists an integer $k$ such that
  \begin{equation*}
    x = k \cdot m + y
  \end{equation*}
\end{defn}
With this definition Xavier and co. then proceeds to build up a
complete library of known algebraic properties for congruence classes
like the one above. As an example we present the addition of
congruence classes:
\begin{verbatim}
  Lemma eqmod_add:
    forall a b c d, eqmod a b -> eqmod c d
      -> eqmod (a + c) (b + d).
  Proof.
    intros a b c d [k1 EQ1] [k2 EQ2]; red.
    subst a; subst c. exists (k1 + k2). ring.
  Qed.
\end{verbatim}
If $a, b$ is a congruence class pair and $c, d$ also is, then the pair
$a+c, b+d$ is an equivalence pair (for a fixed modulo). The proof proceeds by the
well-known method that there must exist a $k_1$ by the first pair such
that $a = k_1 \cdot m + b$. Likewise $c = k_2 \cdot m + d$. Adding
these 2 equations and rearranging yields $a + c = (k_1 + k_2) \cdot m
 + (b + d)$ which is the form one wants. Many of the proofs reflect
usual mathematical methods like this.

\paragraph{The Store}

The JANUS store can be understood as a number of cells, with each cell
storing a number. Cells have \emph{locations}, so we can read from a
cell at a given location or store a value at a given location.

In JANUS, we must take great care when handling the store. It is
paramount that we do not destroy information when we update the
store. For the store itself, however, we won't define it in a way
which protect destruction of information. Rather, we will define a
semantics on top of the store to protect it.

When the store is initialized, every cell will have the value of
$0$. This leads to the straightforward store definition
\begin{equation*}
  \sigma \colon Loc \to W^{32}
\end{equation*}
where we initialize the store to be the function $\sigma(l) = 0$ for
all $l \in Loc$. Reading the store at location $l$ is then simply a
matter of taking the image $\sigma(l)$. Writing to the store is done
by extending the definition of the the function. If we want to write
$v$ at location $l$, we define the new function $\sigma'$ as
\begin{equation*}
  \sigma'(k) = \begin{cases}
    v& \text{if}\quad k = l\\
    \sigma(k)&\text{if}\quad k \neq l
    \end{cases}
\end{equation*}

The method described here is not entirly faithful to the JANUS
language however. A JANUS program, as we shall see, defines the
variables in use in the header. Furthermore, by defining that all
unknown locations maps to $0$, we have lost the ability to
discriminate between having no information about the cell and a cell
with the value of $0$. It will become clear later on that this
discrimination is a desirable feature of our store. This leads to a
store definition as the following:
\newcommand{\lift}[1]{\lfloor #1 \rfloor}
\begin{defn}
  A Store, $\sigma$ is defined to a function $\sigma \colon Loc \to
  W^{32}_\perp$. The image $W^{32}_\perp$ is either the no-information
  value of $\perp$ or it is a normal (lifted) value from $W^{32}$,
  e.g. $\lift{37}$. The store is initialized with $\sigma(l) = \;\perp$
  for all $l \in Loc$.

  To extend the store at location $l$ with value $v$ we use the
  function
  \begin{equation*}
    \sigma'(k) = \begin{cases}
      \lift{v}& \text{if}\quad l = k\\
      \sigma(k)& \text{if}\quad l \neq k
      \end{cases}
  \end{equation*}
\end{defn}
It will be beneficial to our work if we can hide certain values. That
is, we can temporarily define a location $x$ to carry no
information. This is easily achieved with the following definition
\begin{equation*}
  (\sigma \setminus x)(k) = \begin{cases}
    \perp& \text{if}\quad k = x\\
    \sigma(k)& \text{otherwise}
  \end{cases}
\end{equation*}

%%% Local Variables: 
%%% mode: latex
%%% TeX-master: "master"
%%% End: 
