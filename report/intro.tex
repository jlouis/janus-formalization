\chapter{Introduction}

The JANUS programming language were conceived as a student project in
1982 at Caltech by Chris Lutz and Howard Derby. It was first later
however, with the work in \cite{glueck2007} and \cite{glueck2008} that
the language were given a modern treatment. In these papers variants
of the language are established with proper operational semantics. The
results improves on earlier informal descriptions.

Usually when we add formality to a process, there is a need for
change. Formality enforces us to optimize and make precise the
definitions and results, sharpening and refining our knowledge of a
field.

The current state-of-the-art with respect to formality is
machine-verifiable formalizations. We encode our formal work in a way
the machine can understand. We then proceed to encode the meta-theory
about our formalization as well and make the machine verify the
correctness.

The goal of this process is not to prove human beings wrong. It is
rarely the case that the checking process of humans fails to notice
errors in meta-theory. Rather, the goals are those of simplicity,
detailed understanding and domain knowledge:

When we explain a theorem to a computer, it is easiest to use simple
methods. It is more expensive to formalize a complex principle and use
it rather than taking more time and coming up with a simpler
proof. This forces us to use humility and simplicity in our work.

A proof explained to the computer can only draw on knowledge we
already told the computer. The for any proof, we must be able to
explain any detail, as otherwise it will be impossible to
machine-verify. The result is a very detailed understanding of the
area and language.

Finally, we may hope to gain additional knowledge about the domain in
the formalization process. The precision required yield no place for
imprecision and can not lure our intuition down the wrong path. Hence,
we often learn from the formalization process.

This DIKU graduate project establishes a full formalized semantics for
a variant of the JANUS language in the proof assistant Coq. The
main academic contribution of the paper is the use of syntax-driven
semantics for JANUS: Invalid programs are ruled out by having no possible
inference tree. The secondary contribution establihes the machine
verification of forward and backward determinism of the JANUS variant.

The report first describes a toolbox needed for JANUS
formalization. Then it describes the syntax and semantics of JANUS,
extracted from the work in Coq. Finally, we describe the
meta-theoretic results achieved. The report is honest to the
state-of-the-art: any result is mechanically formalized in Coq.

%%% Local Variables: 
%%% mode: latex
%%% TeX-master: "master"
%%% End: 
