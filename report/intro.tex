\chapter{Introduction}

The JANUS programming language were conceived as a student project in
1982 at Caltech by Chris Lutz and Howard Derby. It was first later
however, with the work in \cite{glueck2007} and \cite{glueck2008} that
the language was given a modern treatment. In these papers, variants
of the language are established with proper operational semantics. The
results improves on earlier informal descriptions.

Usually when we add formality to a process, there is a need for
changing the original informal description. Formality enforces us to
optimize and make precise the definitions and results, which in turn
sharpens and refines our knowledge of a field.

The current state-of-the-art with respect to formality is
machine-verifiable formalizations. We encode our formal work in a way
the machine can understand. We then proceed to encode the meta-theory
about our formalization as well. Finally, we ask the machine to
systematically verify the correctness of our proofs.

The goal of this process is not to prove human beings wrong. It is
rarely the case that the checking process of ``the human grid
computer'' fails to notice errors in meta-theory. Rather, the goals
are those of simplicity, detailed understanding and domain knowledge:

When we explain a theorem to a computer, it is easiest to use simple
methods. It is more expensive to formalize a complex principle and use
it rather than taking more time and coming up with a simpler
proof. This forces us to use humility and simplicity in our work.

A proof explained to the computer can only draw on knowledge we
already told the computer. The for any proof, we must be able to
explain any detail, as otherwise it will be impossible to
machine-verify. The result is a very detailed understanding of the
area and language.

Finally, we may hope to gain additional knowledge about the domain in
the formalization process. The precision required yield no place for
imprecision and can not lure our intuition down the wrong path. Hence,
we often learn from the formalization process.

This DIKU graduate project establishes a full formalized semantics for
a variant of the JANUS language in the proof assistant Coq. The
main academic contribution of the paper is the use of syntax-driven
semantics for JANUS: Invalid programs are ruled out by having no possible
inference tree. The secondary contribution establihes the machine
verification of forward and backward determinism of the JANUS variant.

The report first describes a toolbox needed for JANUS
formalization. Then it describes the syntax and semantics of JANUS,
extracted from the work in Coq. Finally, we describe the
meta-theoretic results achieved. The report is honest to the
state-of-the-art: any result is mechanically formalized in Coq.

\section{Why Coq}

One might want to know what decided the formal verification framework
to use. The author has used two frameworks in the past, Coq and Twelf
and it felt logical to use one of those in the formal verification.

One has to make the choice of framework early on and then stick to
it. It is exactly like choosing a programming language for a given
task.

Coq is a proof assistant using an (co-)inductive variant of the
Calculus of Constructions (CoC). It relies on the Curry-Howard
correspondence to prove a link between program/proof and
types/theorems. Proof are carried out by giving tactics. These tactics
then build up the desired proof term and a low level type checker
makes sure the proof is correct.

Twelf uses a dependently typed lambda calculus constrained to certain
canonical forms. This calculus is called LF. The logic and the proofs
are then encoded in this LF-language. A separate checker then verifies
certain properties of the encoding, which finally constitutes a proof
\fixme{Reference to Harper/Crary}.

Coq supports proof automation where parts of proofs are done
automatically by the computer. However, most of the time carrying out
proofs are spent on things other than the proof structure: a
considerable amount of time is spent on figuring out how to condiuct
the proof. Another time consumer is feeding the necessary prior
knowledge to the proof system for the development. You can only carry
out 32-bit addition if the system knows, by definition, what a 32-bit
number is, and what addition means. Properties of JANUS relies on
properties of 32-bit addition, so this must be fed to the framework as
well.

The JANUS development needs 32-bit numbers. These were available
for Coq, but not for Twelf. One could have used classic integers of
infinite size however. Coq further had support for arithmetic in
libraries, whereas in Twelf, we would have to encode this
ourselves. On the other hand, Twelf has a simpler system and excels at
working with lambda-style binders. For Twelf, it would be beneficial
to encode the store as \emph{contextual worlds}. Twelf also has
support for several mutual induction schemes. These are also available
in Coq, but are considerably harder to use. From Coq version 8.2
however, some syntactic sugar has been added to make its use
easier. But at the start of the project, only Coq version 8.1 were
available.


It was deemed that it would be best to have easy access to 32-bit
numbers and hence Coq were chosen. In hindsight certain mutual
induction proofs would probably have been easier to carry out in Twelf
and to rely on arbitrary sized integers would not have been
detrimental to the demonstration of the findings in this report.

\section{Reversible computation}
\label{sec:revers-comp}

Suppose we are given a program $p$ and give it an input $x$ and it
yields an output $y$:
\begin{equation*}
  |[p|](x) = y
\end{equation*}
For some programs $p$ it will be possible to reconstruct $x$ from the
pair $(p, y)$.

To achieve this inversion property certain things must be
fulfilled. The program $p$ must at least be injective: Different
inputs must map the different outputs. Otherwise it is impossible to
reconstruct the input from the output in all cases. As an example,
take the Standard ML program
\begin{verbatim}
  fun even_odd x =
    x mod 2
\end{verbatim}
which will take any number $x$ and return either $0$ or $1$. All even
numbers will be assigned to $0$ and thus it is impossible to know
which even number was the input. The program is not injective and thus
not reversible.

\begin{defn}
  A program $p$ is called \emph{information preserving} if for
  $|[p|](x) = y$ it is possible to uniquely reconstruct $x$ from the
  pair $(p, y)$.
\end{defn}
Informally, no loss of information may occur in the computation. We
are interested in this property as it a necessity for quantum
computing \fixme{CITE QC} and for providing programs for fully
adiabatic CMOS technology \fixme{CITE ADIABATIC}.

\begin{thm}
  Any information preserving program is injective.
\end{thm}
\begin{proof}
  Assume the program is not injective. Then we have $x \neq y$ but
  $|[p|](x) = |[p|](y)$. but then we can't uniquely reconstruct $x$
  say since $y$ is equally likely.
\end{proof}
Injectivity is a necessary condition. The wording of information
preservation is carefully chosen such that there \emph{is} an input
yielding the given output. The output space is then the image of all
possible inputs, automatically making the statement surjective on the
codomain.

For a reversible programming language, like JANUS, we obtain the
information preservation by means of an inversion property: Any
program $p$ can be inverted into a program $p^{-1}$ with
$|[p^{-1}|](y) = x$. Thus we can always reconstruct $x$ from the pair
$(p, y)$ by inverting $p$ and inputting $y$.

JANUS is reversible because it is possible to define a term-rewriting
system for turning a program into its inverse. We will see that it is
possible since each atomic instruction is ivertible and each compound
instruction is invertible if its subcomponents are. This let us use an
inductive argument to show all programs reversible.

Before we discuss the JANUS language itself however, we need to
discuss some components used in JANUS.

%%% Local Variables: 
%%% mode: latex
%%% TeX-master: "master"
%%% End: 
