\chapter{Introduction}

Reversibility is, informally, the research area of computability where
programs can be inverted and run backwards. The most important
side-effect of reversibility is that no information can be lost when
the program executes. This makes reversible programming a venue for
programming CPUs which are fully adiabatic. Such CPUs CMOS technology
will, at least in theory, generate less heat when executing
\fixme{cite}. In this project we study one such reversible programming
language called JANUS. This has been done before, but we add the twist
of formalizing the language in a machine-verifiable form. We thus
construct a bridge between reversibility research and mechanical
theorem proving.

The JANUS programming language were conceived as a student project in
1982 at Caltech by Chris Lutz and Howard Derby. It was first later
however, with the work in \cite{glueck+2007, glueck+2008} that the
language was given a modern formal treatment. In these papers,
variants of the language are established with operational
semantics. This improves the earlier informal description and provides
a vehicle for laying bare the meta-theory of the language. However,
the meta-theory established in these articles are paper-only.

Our motivation for formalizing JANUS inside the machine is that of
reassurance and knowledge. We will be assured that the language has a
working formalization and that key theorems are true. A mechanical
correctness check by a machine is much less amenable to oversigts or a
wrong proof. In fact, if you trust the computation kernel in the
theorem prover of choice, no logical error is possible.

Secondly, it is often the case that a mechanical formalization will
yield new interesting insights in the formalization and proofs. Since
``the devil is in the details'' and we are forced to write down proofs
in great detail, we will often uncover some devils in the
formalization process. Working around the problems with these devils
commonly yields better understanding of the formalization than what we
had before. Formality enforces us to optimize and make precise the
definitions and results, which in turn sharpens and refines our
knowledge of a field. A better understanding through formality makes
it easier to gather new knowlegde in a field.

Finally, the meta-theoretic properties of reversible languages are
rather unique to the field. Hence, the community of mechanical
verification gains a new set of properties to work with.

This DIKU graduate project establishes a formalized semantics for
a variant of the JANUS language in the proof assistant \coq{}. The
main academic contribution of the paper is the use of syntax-driven
semantics for JANUS: Invalid programs are ruled out by having no possible
inference tree. The secondary contribution establishes the machine
verification of the main theorems of JANUS variants.

To make the work digestible, it has been split in several subsets of
the JANUS language. First, we introduce some simplified versions
before processing the complete language. It is our hope that these
subsets make it easier to understand the full development.

The report also describes the toolbox needed for formalization with
the concepts introduced as they are needed.

Finally, this report is honest to the state-of-the-art: any result is
mechanically formalized and verified in \coq{}.

The rest of the paper is structured as follows.
\begin{itemize}
\item In section \ref{sec:revers-comp}, we give a short introduction
  to the concept of reversible computation. We present the general
  ideas and provides formal notions of reversibility.
\item In chapter \ref{chap:thm-prover-choice} we choose a theorem proving
  framework. There are many frameworks one can use for formalization,
  each with their advantages and disadvantages; hence one must be chosen.
\item Then, in chapter \ref{chap:janus0}, we proceed to formalize a
  small subset of JANUS we call \janusz{}. This acts as a driver for
  introducing the foundations of the formalizations while keeping the
  language simple and lean.
\item Chapter \ref{chap:janus1} layers new concepts on top of
  \janusz{}. 32-bit unsigned integers are introduced in the
  formalization, procedure calls are added. The resulting language,
  called \januso{} is presented formally.
\item Then, in chapter \ref{chap:fulljanus} we tackle the remaining
  parts of the JANUS-language variant we have decided to work with. We
  point to various problems with the formalization of full JANUS and
  we give ideas for solving these.
\item The report wraps up and closes with a discussion of further work
  and a conclusion.
\end{itemize}

\subsection{Related work}

There are several examples of reversible programming languages:
reversible Turing machines \cite{morita+2007:reversible-tm},
functional languages \cite{glueck+2005:program-inverter} etc.. There
are also many examples of programming language formalizations with a
large diversity: functional languages
\cite{leroy+2009:coinductive-big-step}, substructural languages
\cite{fluett+2007:phd-thesis} and foundational languages
\cite{jlouis+2007:bachelor}

To our knowledge, nobody has tried to formalize reversible programming
languages using mechanical verification tools before this project. As
such, this project combines two otherwise unrelated fields: namely the
field of mechanical verification of programming language properties
and the field of reversible language theory.

\subsection{Expectations}

We expect the reader to be well versed in the concepts of programming
language semantics and proofs of programming language meta-theory. We
further expect the reader to know the general idea of logical
frameworks, proof assistants and theorem provers. We won't describe
the concepts of \coq{} in much detail and just point to the literature
\cite{bertot+2004:coq-art}.

We also expect the reader to be somewhat familiar with the concept of
reversible languages. In particular we expect the reader to know what
it means for a language to be reversible in the informal sense and we
expect the reader to know, in general, the semantics of JANUS.

\section{Reversible computation}
\label{sec:revers-comp}

\fixme{clean up this section}
We now give a short introduction to the concept of reversible
computation: suppose we are given a program $p$ and give it an input
$x$ and it yields an output $y$:
\begin{equation*}
  |[p|](x) = y
\end{equation*}
For some programs $p$ it will be possible to reconstruct $x$ from the
pair $(p, y)$.

To achieve this inversion property certain things must be
fulfilled. The program $p$ must at least be injective: Different
inputs must map the different outputs. Otherwise it is impossible to
reconstruct the input from the output in all cases. As an example,
take the Standard ML program
\begin{verbatim}
  fun even_odd x =
    x mod 2
\end{verbatim}
which will take any number $x$ and return either $0$ or $1$. All even
numbers will be assigned to $0$ and thus it is impossible to know
which even number was the input. The program is not injective and thus
not reversible.

\begin{defn}
  A program $p$ is called \emph{information preserving} if for
  $|[p|](x) = y$ it is possible to uniquely reconstruct $x$ from the
  pair $(p, y)$.
\end{defn}
Informally, no loss of information may occur in the computation.

\begin{thm}
  Any information preserving program is injective.
\end{thm}
\begin{proof}
  Assume the program is not injective. Then we have $x \neq y$ but
  $|[p|](x) = |[p|](y)$. but then we can't uniquely reconstruct $x$
  say since $y$ is equally likely.
\end{proof}
Injectivity is a necessary condition. The wording of information
preservation is carefully chosen such that there \emph{is} an input
yielding the given output. The output space is then the image of all
possible inputs, automatically making the statement surjective on the
co-domain.

For a reversible programming language, like JANUS, we obtain the
information preservation by means of an inversion property: Any
program $p$ can be inverted into a program $p^{-1}$ with
$|[p^{-1}|](y) = x$. Thus we can always reconstruct $x$ from the pair
$(p, y)$ by inverting $p$ and inputting $y$.

JANUS is reversible because it is possible to define a term-rewriting
system for turning a program into its inverse. We will see that it is
possible since each atomic instruction is invertible and each compound
instruction is invertible if its components are. This let us use an
inductive argument to show all programs reversible.

%%% Local Variables: 
%%% mode: latex
%%% TeX-master: "master"
%%% End: 
